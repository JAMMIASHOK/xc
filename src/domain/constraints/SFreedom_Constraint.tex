% File: ~/domain/constraints/SFreedom\_Constraint.tex 

\noindent {\bf Files}
\indent \#include \f$<\tilde{ }\f$/domain/constraints/SFreedom\_Constraint.h\f$>\f$

\noindent {\bf Class Declaration}
\indent class SFreedom\_Constraint: public DomainComponent

\noindent {\bf Class Hierarchy}
\indent TaggedObject
\indent MovableObject
\indent\indent DomainComponent
\indent\indent\indent {\bf SFreedom\_Constraint}

\noindent {\bf Description}
\indent An SFreedom\_Constraint represents a single point constraint in the
//! domain. A single point constraint specifies the response of a particular
//! degree-of-freedom at a node. The declaration that all methods are
//! virtual allows for time varying constraints to be introduced.

\noindent {\bf Class Interface}
\indent\indent // Constructors
\indent\indent {\em SFreedom\_Constraint(int tag, int nodeTag, int ndof,
//! double value, bool isConstant = true);}
\indent\indent {\em SFreedom\_Constraint(int tag, int nodeTag, int ndof, int classTag);} 
\indent\indent {\em SFreedom\_Constraint(int classTag);}
\indent\indent // Destructor
\indent\indent {\em virtual \f$\tilde{ }\f$ SFreedom\_Constraint();}
\indent\indent // Public Methods
\indent\indent {\em virtual int getNodeTag(void) const;}
\indent\indent {\em virtual int getDOF\_Number(void) const;}
\indent\indent {\em virtual int applyConstraint(double loadFactor)
//! const;} 
\indent\indent {\em virtual double getValue(void) const;}
\indent\indent {\em virtual bool isHomogeneous(void) const;} 
\indent\indent {\em virtual void setLoadPatternTag(int loadPaternTag);}
\indent\indent {\em virtual int  getLoadPatternTag(void) const;}
\indent\indent // Public Methods for Output
\indent\indent {\em virtual int sendSelf(int commitTag, Channel \&theChannel);} 
\indent\indent {\em virtual int recvSelf(int commitTag, Channel \&theChannel,
//! FEM\_ObjectBroker \&theBroker);} 
\indent\indent {\em virtual void Print(OPS_Stream \&s, int flag = 0);}


\noindent {\bf Constructors}
\indent {\em SFreedom\_Constraint(int tag, int nodeTag, int ndof, double value);}
//! To construct a single point constraint to constrain the trial
//! displacement of the \p ndof'th dof at node \p node to the value
//! given by \p value. The integer value \p tag is used to identify
//! the SFreedom\_Constraint among all other SFreedom\_Constraints. If
\p value is equal to \f$0.0\f$ the method isHomogeneous() will
//! always return \p true, otherwise \p false. 

\indent {\em SFreedom\_Constraint(int tag, int node, int ndof, int classTag);}
//! Provided for subclasses to use. The subclasses can vary the value of the
//! imposed displacement when getValue() is invoked. If this
//! constructor is used the isHomogeneous() method will always
//! return \p false. The integer value \p tag is used to identify
//! the SFreedom\_Constraint among all other SFreedom\_Constraints.


\indent {\em SFreedom\_Constraint(int classTag);}
//! Provided for the FEM\_ObjectBroker to be able to instantiate an
//! object; the data for this object will be read from a Channel object
//! when recvSelf() is invoked. \f$0\f$ and \p classTag are passed to
//! the DomainComponent constructor.


\noindent {\bf Destructor}
\indent {\em virtual \f$\tilde{ }\f$ SFreedom\_Constraint();}
//! Does nothing. Provided so that a subclasses destructor can be
//! invoked.


\noindent {\bf Public Methods }
\indent {\em virtual int getNodeTag(void) const;}
//! Returns the value of \p node passed in the constructor, this should be 
//! the tag of the node that is being constrained.

{\em virtual int getDOF\_Number(void) const;}
//! Returns the value of \p ndof that was passed in the constructor,
//! this identifies the dof number corresponding to the constraint.

\indent {\em virtual int applyConstraint(double loadFactor);} 
//! To set the value of the constraint for the load factor given by {\em
//! loadFactor}. The constraint is set equal to \p loadFactor * {\em
//! value} if the constraint is not constant, or \p value if the
//! constraint was identified as constant in the constructor.

\indent {\em virtual bool isHomogeneous(void) const;}
//! To return a boolean indicating whether or not the constraint is a
//! homogeneous constraint. A homogeneous constraint is one where the value
//! of the constraint, \p value, is always \f$0\f$. This information can be used by the
//! ConstraintHandler to reduce the number of equations in the system.

{\em virtual double getValue(void) const;}
//! To return the value of the constraint determined in the last call to
//! applyConstraint(). This base class returns \p value passed in
//! the constructor. 

\indent {\em virtual void setLoadPatternTag(int loadPaternTag);}
//! To set the LoadPattern tag associated with the object to be {\em
//! loadPatternTag}.

\indent {\em virtual int  getLoadPatternTag(void) const;} 
//! To return the load pattern tag associated with the load.

{\em virtual int sendSelf(int commitTag, Channel \&theChannel);} 
//! Creates a Vector, and stores the SFreedom\_Constraints tag, nodeTag, ndof and value in
//! the Vector. It then passes the Vector as an argument to {\em
//! theChannel} objects sendVector() method, along with the objects 
//! database tag and \p commitTag. Subclasses must invoke this method
//! in their implementation of sendSelf(), so that the \p node
//! and \p ndof values in remote object can be set. Returns \f$0\f$ if
//! successful, a negative number if the Channel object, \p theChannel,
//! failed to send the data. 

{\em virtual int recvSelf(int commitTag, Channel \&theChannel, FEM\_ObjectBroker
\&theBroker);} 
//! Creates a Vector, and receives the Vector from the channel object
//! using the recvVector() method call and the objects own database
//! tag and \p commitTag. Using the information contained in the Vector, the 
//! SFreedom\_Constraints tag, nodeTag, ndof and value are set. Subclasses must
//! invoke this method in their implementation of recvSelf(), so
//! that the \p node and \p ndof values can be set. Returns \f$0\f$ if
//! successful, a negative number if the Channel object, {\em
//! theChannel}, failed to receive the data.   

{\em virtual void Print(OPS_Stream \&s, int flag = 0) const;}
//! Prints out the SFreedom\_Constraints tag, then \p node, \p ndof and
\p value. 

