% File: ~/domain/constraints/SFreedom\_Constraint.tex 

\noindent {\bf Files}   \\
\indent \#include $<\tilde{ }$/domain/constraints/SFreedom\_Constraint.h$>$  \\

\noindent {\bf Class Declaration}  \\
\indent class SFreedom\_Constraint: public DomainComponent \\

\noindent {\bf Class Hierarchy} \\
\indent TaggedObject \\
\indent MovableObject \\
\indent\indent DomainComponent \\
\indent\indent\indent {\bf SFreedom\_Constraint} \\

\noindent {\bf Description}  \\
\indent An SFreedom\_Constraint represents a single point constraint in the
domain. A single point constraint specifies the response of a particular
degree-of-freedom at a node. The declaration that all methods are
virtual allows for time varying constraints to be introduced. \\

\noindent {\bf Class Interface}  \\
\indent\indent // Constructors  \\
\indent\indent {\em SFreedom\_Constraint(int tag, int nodeTag, int ndof,
double value, bool isConstant = true);} \\
\indent\indent {\em SFreedom\_Constraint(int tag, int nodeTag, int ndof, int classTag);}\\ 
\indent\indent {\em SFreedom\_Constraint(int classTag);} \\ \\
\indent\indent // Destructor  \\
\indent\indent {\em virtual $\tilde{ }$ SFreedom\_Constraint();} \\ \\
\indent\indent // Public Methods  \\
\indent\indent {\em virtual int getNodeTag(void) const;} \\
\indent\indent {\em virtual int getDOF\_Number(void) const;} \\
\indent\indent {\em virtual int applyConstraint(double loadFactor)
const;} \\ 
\indent\indent {\em virtual double getValue(void) const;} \\
\indent\indent {\em virtual bool isHomogeneous(void) const;}\\ 
\indent\indent {\em virtual void setLoadPatternTag(int loadPaternTag);}\\
\indent\indent {\em virtual int  getLoadPatternTag(void) const;}\\ \\
\indent\indent // Public Methods for Output \\
\indent\indent {\em virtual int sendSelf(int commitTag, Channel \&theChannel);} \\ 
\indent\indent {\em virtual int recvSelf(int commitTag, Channel \&theChannel,
FEM\_ObjectBroker \&theBroker);} \\ 
\indent\indent {\em virtual void Print(OPS_Stream \&s, int flag = 0);} \\


\noindent {\bf Constructors}  \\
\indent {\em SFreedom\_Constraint(int tag, int nodeTag, int ndof, double value);} \\
To construct a single point constraint to constrain the trial
displacement of the {\em ndof}'th dof at node {\em node} to the value
given by {\em value}. The integer value {\em tag} is used to identify
the SFreedom\_Constraint among all other SFreedom\_Constraints. If
{\em value} is equal to $0.0$ the method {\em isHomogeneous()} will
always return {\em true}, otherwise {\em false}. \\ 

\indent {\em SFreedom\_Constraint(int tag, int node, int ndof, int classTag);} \\
Provided for subclasses to use. The subclasses can vary the value of the
imposed displacement when {\em getValue()} is invoked. If this
constructor is used the {\em isHomogeneous()} method will always
return {\em false}. The integer value {\em tag} is used to identify
the SFreedom\_Constraint among all other SFreedom\_Constraints. \\


\indent {\em SFreedom\_Constraint(int classTag);} \\
Provided for the FEM\_ObjectBroker to be able to instantiate an
object; the data for this object will be read from a Channel object
when {\em recvSelf()} is invoked. $0$ and {\em classTag} are passed to
the DomainComponent constructor. \\


\noindent {\bf Destructor}  \\
\indent {\em virtual $\tilde{ }$ SFreedom\_Constraint();} \\
Does nothing. Provided so that a subclasses destructor can be
invoked. \\


\noindent {\bf Public Methods }  \\
\indent {\em virtual int getNodeTag(void) const;} \\
Returns the value of {\em node} passed in the constructor, this should be 
the tag of the node that is being constrained. \\

{\em virtual int getDOF\_Number(void) const;} \\
Returns the value of {\em ndof} that was passed in the constructor,
this identifies the dof number corresponding to the constraint. \\

\indent {\em virtual int applyConstraint(double loadFactor);}\\ 
To set the value of the constraint for the load factor given by {\em
loadFactor}. The constraint is set equal to {\em loadFactor} * {\em
value} if the constraint is not constant, or {\em value} if the
constraint was identified as constant in the constructor. \\

\indent {\em virtual bool isHomogeneous(void) const;}\\
To return a boolean indicating whether or not the constraint is a
homogeneous constraint. A homogeneous constraint is one where the value
of the constraint, {\em value}, is always $0$. This information can be used by the
ConstraintHandler to reduce the number of equations in the system. \\

{\em virtual double getValue(void) const;} \\
To return the value of the constraint determined in the last call to
{\em applyConstraint()}. This base class returns {\em value} passed in
the constructor.  \\ 

\indent {\em virtual void setLoadPatternTag(int loadPaternTag);}\\
To set the LoadPattern tag associated with the object to be {\em
loadPatternTag}. \\

\indent {\em virtual int  getLoadPatternTag(void) const;}\\ 
To return the load pattern tag associated with the load. \\

{\em virtual int sendSelf(int commitTag, Channel \&theChannel);} \\ 
Creates a Vector, and stores the SFreedom\_Constraints tag, nodeTag, ndof and value in
the Vector. It then passes the Vector as an argument to {\em
theChannel} objects {\em sendVector()} method, along with the objects 
database tag and {\em commitTag}. Subclasses must invoke this method
in their implementation of {\em sendSelf()}, so that the {\em node}
and {\em ndof} values in remote object can be set. Returns $0$ if
successful, a negative number if the Channel object, {\em theChannel},
failed to send the data. \\ 

{\em virtual int recvSelf(int commitTag, Channel \&theChannel, FEM\_ObjectBroker
\&theBroker);} \\ 
Creates a Vector, and receives the Vector from the channel object
using the {\em recvVector()} method call and the objects own database
tag and {\em commitTag}. Using the information contained in the Vector, the 
SFreedom\_Constraints tag, nodeTag, ndof and value are set. Subclasses must
invoke this method in their implementation of {\em recvSelf()}, so
that the {\em node} and {\em ndof} values can be set. Returns $0$ if
successful, a negative number if the Channel object, {\em
theChannel}, failed to receive the data. \\   

{\em virtual void Print(OPS_Stream \&s, int flag = 0) const;} \\
Prints out the SFreedom\_Constraints tag, then {\em node}, {\em ndof} and
{\em value}. 

