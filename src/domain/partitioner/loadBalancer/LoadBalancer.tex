% File: ~/domain/domain/loadBalancer/LoadBalancer.tex 
% What: "@(#) LoadBalancer.tex, revA"

\noindent {\bf Files}
\indent \#include \f$<\tilde{ }\f$/domain/loadBalancer/LoadBalancer.h\f$>\f$

\noindent {\bf Class Decleration}
\indent class LoadBalancer

\noindent {\bf Class Hierarchy}
\indent {\bf LoadBalancer}


\noindent {\bf Description}
\indent A LoadBalancer is an object used to balance a
//! PartitionedDomain. The LoadBalancer does this by invoking methods in
//! the DomainPartitioner object with which it is associated.

\noindent {\bf Class Interface}
\indent\indent  // Constructor
\indent\indent {\em LoadBalancer(); }
\indent\indent // Destructor
\indent\indent {\em virtual~ \f$\tilde{}\f$LoadBalancer();}
\indent\indent // Public Methods
\indent\indent {\em virtual void setLinks(DomainPartitioner
\&thePartitioner);} 
\indent\indent {\em virtual int balance(Graph \&theWeightedGraph) =0;}
\indent\indent // Public Methods
\indent\indent {\em DomainPartitioner *getDomainPartitioner(void); }


\noindent {\bf  Constructor  }
\indent {\em LoadBalancer(); } 
//! Sets the pointer to the assocaited PartitionedDomain to be \f$0\f$.

\noindent {\bf Destructor }
\indent {\em virtual~ \f$\tilde{}\f$LoadBalancer();} 
//! Does nothing. Provided so the subclasses destructor will be called.

\noindent {\bf  Public Methods}
\indent {\em virtual void setLinks(DomainPartitioner
\&thePartitioner);}
//! Sets the pointer to the DomainPartitioner object associated with the
//! LoadBalancer to point to \p thePartitioner.

{\em virtual int balance(Graph \&theWeightedGraph) =0;} 
//! Each subclass must provide an implementation of this method. This
//! method is invoked to balance the PartitionedDomain. The Graph {\em
//! theWeightedGraph} is a weighted graph in which the vertices represent
//! the Subdomains, the edges Subdomains sharing common boundary nodes and
//! the weight the cost of the previous Subdomain calculations (determined
//! by invoking getCost() on the Subdomains).


\noindent {\bf Protected Methods}
\indent {\em DomainPartitioner *getDomainPartitioner(void); }
//! Returns a pointer to the DomainPartitioner. If no DomainPartitioner
//! has been set, a warning message is printed and \f$0\f$ returned.




