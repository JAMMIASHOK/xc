% File: ~/domain/domain/Domain.tex 
% What: "@(#) Domain.tex, revA"

\noindent {\bf Files}   
\indent \#include $<$/domain/domain/Domain.h$>$  

\noindent {\bf Class Declaration}  
\indent class Domain  

\noindent {\bf Class Hierarchy} 
\indent  {\bf Domain} 

\noindent {\bf Description}  

\noindent {\bf Constructors}  
\indent {\em Domain();}  
\   

\indent {\em Domain(int numNodes, int numElements, int numSPs, int
//! numMPs, int numLoadPatterns);}   
//!    


\indent {\em Domain(TaggedObjectStorage \&theNodesStorage, 
\indent\indent\indent           TaggedObjectStorage \&theElementsStorage, 
\indent\indent\indent           TaggedObjectStorage \&theMPsStorage, 
\indent\indent\indent           TaggedObjectStorage \&theSPsStorage, 
\indent\indent\indent           TaggedObjectStorage \&theLoadPatternsStorage); } 
//! Constructs a Domain where the Nodes, Elements, MP\_Constraints,
//! SFreedom\_Constraint and LoadPattern objects will be
//! stored in the storage objects provided. A check is made to ensure
//! these container objects are empty is made; if not empty the warning()
//! method of the global ErrorHandler is invoked and the objects are
//! cleared. 

\indent {\em Domain(TaggedObjectStorage \&theStorageType); } 
//! Constructs a Domain where the Nodes, Elements, MP\_Constraints,
//! SFreedom\_Constraint and LoadPattern objects will be stored in the storage
//! objects obtained by invoking getEmptyCopy() on the {\em
//! theStorageType} object. A check is made to ensure that memory is
//! allocated for these objects, if not the fatal() method of the
//! global ErrorHandler is invoked.    

\noindent {\bf Destructor}  
\indent {\em virtual~ $\tilde{}$Domain();}  
//! Invokes delete on all the storage objects. This means that, if the two
//! latter constructors have been called, the container objects must have 
//! been created using \p new and that at no other point in the program
//! is the destructor invoked on these objects. It should be noted, that
//! the objects in the Domain, i.e. the DomainComponents, are not
//! destroyed. To clean up these objects clearAll() should be
//! invoked before the destructor is called. 

%%%%%%%% Public Member Functions - POPULATION
\noindent {\bf Public Methods to Populate the Domain}  
\indent {\em virtual bool addElement(Element *theElementPtr);}  

{\em virtual bool addNode(Node *theNodePtr = false);}  
//! 


{\em virtual bool addSP(SFreedom\_Constraint *theSPptr = false);}  




{\em virtual bool addMP(MP\_Constraint *theMPptr = false);}  
//! To add the multiple point constraint pointed to by theMPptr, to the
//! domain. If {\em \_DEBUG} is defined the domain first
//! checks to see that the retained and the constrained node both exist
//! in the model and that the matrix is of proper dimensions (THIS LAST
//! PART NOT YET IMPLEMENTED). 
//! In addition the domain always checks to ensure that no other
//! MP\_Constraint with a similar tag exists in the domain. If the checks are
//! successful, the constraint is added to domain by the domain invoking {\em
//! addComponent(theMPptr)} on the container for the MP\_Constraints. The
//! domain then invokes {\em setDomain(this)} on the 
//! constraint and domainChange() on itself. The call returns {\em
//! true} if the constraint was added, otherwise the warning() method of
//! the global ErrorHandler is invoked and \p false is returned.


{\em virtual bool addLoadPattern(LoadPattern *thePattern);}   
//! To add the LoadPattern \p thePattern to the domain.
//! The load is added to domain by the domain invoking {\em
//! addComponent(theLd)} on the container for the LoadPatterns. The domain
//! is responsible for invoking {\em setDomain(this)} on the load. The
//! call returns \p true if the load was added, otherwise the {\em
//! warning()} method of the global ErrorHandler is invoked and {\em
//! false} is returned. 

{\em virtual bool addNodalLoad(NodalLoad *theLd, int loadPatternTag);}   
//! To add the nodal load \p theld to the LoadPattern in the domain
//! whose tag is given by \p loadPatternTag.
//! If {\em \_DEBUG} is defines, checks to see that corresponding node
//! exists in the domain. A pointer to the LoadPattern is obtained from
//! the LoadPattern container and the load is added to LoadPattern by
//! invoking {\em addNodalLoad(theLd)} on the LoadPattern object. The
//! domain is responsible for invoking {\em setDomain(this)} on the
//! load. The call returns \p true if the load was added, otherwise the
//! warning() method on the global ErrorHandler is invoked and {\em
//! false} is returned. 


{\em virtual bool addElementalLoad(ElementalLoad *theLd, int loadPatternTag);}   
//! To add the elemental load \p theld to the LoadPattern in the domain
//! whose tag is given by \p loadPatternTag.
//! If {\em \_DEBUG} is defines, checks to see that corresponding element
//! exists in the domain. A pointer to the LoadPattern is obtained from
//! the LoadPattern container and the load is added to LoadPattern by
//! invoking {\em addElementalLoad(theLd)} on the LoadPattern object. The
//! domain is responsible for invoking {\em setDomain(this)} on the
//! load. The call returns \p true if the load was added, otherwise the
//! warning() method on the global ErrorHandler is invoked and {\em
//! false} is returned. 


{\em virtual bool addSFreedom\_Constraint(SFreedom\_Constraint *theConstraint, int
//! loadPatternTag);}    
//! To add the elemental load \p theConstraint to the LoadPattern in the domain
//! whose tag is given by \p loadPatternTag.
//! If {\em \_DEBUG} is defines, checks to see that corresponding node
//! exists in the domain. A pointer to the LoadPattern is obtained from
//! the LoadPattern container and the load is added to LoadPattern by
//! invoking {\em addSFreedom\_Constraint(theConstraint)} on the LoadPattern object. The
//! domain is responsible for invoking {\em setDomain(this)} on the
//! constraint. The call returns \p true if the load was added, otherwise the
//! warning() method on the global ErrorHandler is invoked and {\em
//! false} is returned. 


\noindent {\bf Public Methods to Depopulate the Domain}  
\indent{\em virtual void clearAll(void);}
//! To remove all the components from the model and invoke the destructor
//! on these objects. First clearAll() is invoked on all the
//! LoadPattern objects. Then the domain object invokes {\em
//! clearAll()} on its container objects. In addition the destructor for
//! each Recorder object that has been added to the domain is invoked. In
//! addition the current time and load factor are set to $0$, and the box
//! bounding the domain is set to the box enclosing the origin. 


\indent{\em virtual Element *removeElement(int tag);}
//! To remove the element whose tag is given by \p tag from the
//! domain. The domain achieves this by invoking {\em
//! removeComponent(tag)} on the container for the elements. 
//! Returns $0$ if no such element exists in the domain. Otherwise 
//! the domain invokes {\em setDomain(0)} on the element (using a cast to
//! go from a TaggedObject to an Element, which is safe as only an
//! Element objects are added to this container) and {\em
//! domainChange()} on itself before a pointer to the element is returned. 

{\em virtual Node *removeNode(int tag);}    
//! To remove the node whose tag is given by \p tag from the domain. 
//! The domain achieves this by invoking {\em
//! removeComponent(tag)} on the container for the nodes. 
//! Returns $0$ if no such node exists in the domain. If the node is to be
//! removed the domain invokes {\em setDomain(0)} on the node and {\em
//! domainChange()} on itself before a pointer to the Node is returned. 

{\em virtual SFreedom\_Constraint *removeSFreedom\_Constraint(int tag);}
//! To remove the SFreedom\_Constraint whose tag is given by \p tag from the
//! domain. The domain achieves this by invoking {\em
//! removeComponent(tag)} on the container for the single point
//! constraints. Returns $0$ if the constraint was not in the domain,
//! otherwise the domain invokes {\em setDomain(0)} on the constraint and
//! domainChange() on itself before a pointer to the constraint is
//! returned. Note this will only remove SFreedom\_Constraints which have been
//! added to the domain and not directly to LoadPatterns.

{\em virtual MP\_Constraint *removeMP\_Constraint(int tag);} 
//! To remove the MP\_Constraint whose tag is given by \p tag from the
//! domain. The domain achieves this by invoking {\em
//! removeComponent(tag)} on the container for the multi point
//! constraints. Returns $0$ if the constraint was not in the domain,
//! otherwise the domain invokes {\em setDomain(0)} on the constraint and
//! domainChange() on itself before a pointer to the constraint is
//! returned.   

{\em virtual LoadPattern *removeLoadPattern(int tag);}         
//! To remove the LoadPattern whose tag is given by \p tag from the
//! domain. The domain achieves this by invoking {\em
//! removeComponent(tag)} on the container for the load patterns. 
//! If the LoadPattern exists, the domain then iterates through the loads
//! and constraints of the LoadPattern invoking {\em setDomain(0)} on
//! these objects. Returns
$0$ if the load was not in the domain, otherwise returns a pointer to
//! the load that was removed. Invokes {\em setDomain(0)} on the load case
//! before it is returned. 


%%%%%%%% Public Member Functions - ACCESS
\noindent {\bf Public Methods to Access the Components of the Domain}  
\indent {\em virtual ElementIter \&getElements(void);} 
//! To return an iter for the Elements in the domain. It returns an {\em
//! ElementIter} for the elements of the domain that were added using {\em
//! addElement()}.   

{\em virtual NodeIter \&getNodes(void);} 
//! It returns a \p NodeIter for the nodes which have been added to the
//! domain.  

{\em virtual SFreedom\_ConstraintIter \&getSPs(void);} 
//! To return an {\em SFreedom\_ConstraintIter} for the single point constraints
//! which have been added to the domain.  

{\em virtual MP\_ConstraintIter \&getMPs(void);} 
//! To return an {\em MP\_ConstraintIter} for the multiple point
//! constraints which have been added to the domain.  

{\em virtual LoadPatternIter \&getLoadPatterns(void);} 
//! To return an \p LoadPatternIter for the LoadPattern
//! objects which have been added to the domain.  

{\em virtual  Element *getElement(int tag);}
//! To return a pointer to the element \p tag. If no such element
//! exists $0$ is returned. It does this by invoking {em
//! getComponentPtr(tag)} on the element container and performing a cast
//! to an element if the object exists. 

{\em virtual  Node *getNode(int tag);}
//! To return a pointer to the node whose tag is given by \p tag. If
//! no such node exists $0$ is returned. It does this by invoking {em
//! getComponentPtr(tag)} on the node container and performing a cast
//! to a node if the object exists. 

{\em virtual  SFreedom\_Constraint *getSFreedom\_ConstraintPtr(int tag);}
//! To return a pointer to the SFreedom\_Constraint whose tag is given by \p tag. If
//! no such SFreedom\_Constraint exists $0$ is returned. It does this by invoking {em
//! getComponentPtr(tag)} on the single-point constraint container and
//! performing a cast to an SFreedom\_Constraint if the object exists. 


{\em virtual  MP\_Constraint *getMP\_ConstraintPtr(int tag);}
//! To return a pointer to the MP\_Constraint whose tag is given by \p tag. If
//! no such MP\_Constraint exists $0$ is returned. It does this by invoking {em
//! getComponentPtr(tag)} on the multi-point constraint container and
//! performing a cast to an MP\_Constraint if the object exists. 


{\em virtual  ElementalLoad *getLoadPattern(int tag);}
//! To return a pointer to the LoadPattern whose tag is given by \p tag. If
//! no such LoadPattern exists $0$ is returned. It does this by invoking {em
//! getComponentPtr(tag)} on the elemental load container and
//! performing a cast to a LoadPattern if the object exists. 


%%%%%%%% Public Member Functions - QUERY
\noindent {\bf Public Methods to Query the Domain} 
\indent {\em virtual int getNumElements(void) const;}
//! To return the number of elements in the domain. It does this by
//! invoking getNumComponents() on the container for the elements. 

{\em virtual int getNumNodes(void) const;}
//! To return the number of nodes in the domain. It does this by
//! invoking getNumComponents() on the container for the
//! nodes. 

{\em virtual int getNumSPs(void) const;}
//! To return the number of single point constraints in the domain. It
//! does this by invoking getNumComponents() on the container for
//! the single point constraints. 

{\em virtual int getNumMPs(void) const;}
//! To return the number of multi point constraints in the domain. It
//! does this by invoking getNumComponents() on the container for
//! the multi point constraints. 

{\em virtual int getNumLoadPatterns(void) const;}
//! To return the number of load patterns in the domain. It
//! does this by invoking getNumComponents() on the container for
//! the load patterns. 

{\em virtual const Vector \&getPhysicalBounds(void);} 
//! To return the bounding rectangle for the Domain. The information is
//! contained in a Vector of size 6 containing in the following order
\{xmin, ymin, zmin, xmax, ymax, zmax\}. This information is built up
//! as nodes are added to the domain, initially all are set to $0$ in the
//! constructor. 

{\em virtual Graph \&getElementGraph(void);} 
//! Returns the current element graph (the connectivity of the elements
//! in the domain). If the \p eleChangeFlag has been set
//! to \p true the method will invoke {\em buildEleGraph(theEleGraph)}
//! on itself before returning the graph. The vertices in the element
//! graph are to be labeled $0$ through $numEle-1$. The Vertices references
//! contain the elemental tags.  

{\em virtual Graph \&getNodeGraph(void);} 
//! Returns the current node graph (the connectivity of the nodes in
//! the domain). If the \p nodeChangeFlag has been set to \p true the
//! will invoke {\em buildNodeGraph(theNodeGraph)} on itself before
//! returning the graph. The vertices in the node graph are to be labeled
$0$ through $numNode-1$. The Vertices references contain the nodal tags.  


%%%%%%%% Public Member Functions - UPDATE
\noindent {\bf Public Methods to Update the Domain}  
\indent {\em virtual setCommitTag(int newTag);} 
//! To set the current commitTag to \p newTag. 

\indent {\em virtual setCurrentTime(double newTime);} 
//! To set the current time to \p newTime. 

\indent {\em virtual setCommittedTime(double newTime);} 
//! To set the committed time to \p newTime. 

\indent {\em virtual void applyLoad(double pseudoTime);}  
//! To apply the loads for the given time \p pseudoTime. The domain
//! first clears the current load at all nodes and elements, done by
//! invoking zeroUnbalancedLoad()} on each node and {\em zeroLoad()
//! on each element. The domain then invokes {\em applyLoad(pseudoTime)}
//! on all LoadPatterns which have been added to the domain. The domain
//! will then invoke {\em applyConstraint(pseudoTime)} on all the
//! constraints in the single and multi point constraint
//! containers. Finally the domain sets its current time to be {\em
//! pseudoTime}.  

\indent {\em virtual setLoadConstant(void);} 
//! To set the loads in the LoadPatterns to be constant. The domain
//! achieves this by invoking setLoadConstant() on all the
//! LoadPatterns which have been added to the domain. Note that
//! LoadPatterns added after this method has been invoked will not be
//! constant until this method is invoked again. 


{\em virtual void commit(void);} 
//! To commit the state of the domain , that is to accept the current
//! state as being ion the solution path. The domain invokes {\em
//! commit()} on all nodes in the domain and then {\em 
//! commit()} on all elements of the domain. These are calls for the nodes
//! and elements to set there committed state as given by their current
//! state. The domain will then set its committed time variable to be
//! equal to the current time and lastly increments its commit tag by $1$.  


{\em virtual int revertToLastCommit(void);} 
//! To return the domain to the state it was in at the last commit. The
//! domain invokes revertToLastCommit() on all nodes and elements in
//! the domain. The domain sets its current loadFactor and time
//! stamp to be equal to the last committed values. The domain decrements
//! the current commitTag by $1$. Finally it invokes applyLoad()
//! on itself with the current time.

{\em virtual void update(void);} 
//! Called by the AnalysisModel to update the state of the
//! domain. Iterates over all the elements of the Domain and invokes {\em
//! update()}. 

\indent {\em virtual void setDomainChangeStamp(int newStamp);}
//! To set the domain stamp to be \p newStamp. Domain stamp is the
//! integer returned by hasDomainChanged(). 

\noindent {\bf Public Methods for the Analysis} 
{\em virtual int hasDomainChanged(void);} 
//! To return an integer stamp indicating the state of the
//! domain. Initially $0$, this integer is incremented by $1$ if  {\em
//! domainChange()} has been invoked since the last invocation of the
//! method. If the domain has changed it marks the element and node graph
//! flags as not having been built.  

\noindent {\bf Public Methods for Output} 
\indent {\em  virtual int  addRecorder(Recorder \&theRecorder);}
//! To add a recorder object \p theRecorder to the domain. {\em
//! record(commitTag)} is invoked on each commit(). The pointers to
//! the recorders are stored in an array which is resized on each
//! invocation of this method.  

\indent {\em int playback(int commitTag);}
//! The domain will invoke {\em playback(commitTag)} on all recorder
//! objects which have been added to the domain.

\indent {\em virtual void Print(OPS_Stream \&s, int flag =0);}
//! To print the state of the domain. The domain invokes {\em Print(s,flag)} on
//! all it's container objects. 


\indent {\em friend OPS_Stream \&operator$<<$(OPS_Stream \&s, Domain \&M); }  
//! This function allows domain objects to be printed to streams. The
//! function invokes $M.Print(s)$ before returning $s$. 

\noindent {\bf Protected Methods}  
\indent{\em virtual void domainChange(void) }
//! Sets a flag indicating that the integer returned in the next call to 
//! hasDomainChanged() must be incremented by $1$. This method is
//! invoked whenever a Node, Element or Constraint object is added to the
//! domain.  

{\em virtual int buildEleGraph(Graph \&theEleGraph)} 
//! A method which will cause the domain to discard the current element
//! graph and build a new one based on the element connectivity. Returns
$0$ if successful otherwise $-1$ is returned along with an error
//! message to opserr. 

{\em virtual int buildNodeGraph(Graph \&theNodeGraph)} 
//! A method which will cause the domain to discard the current node
//! graph and build a new one based on the node connectivity. Returns
$0$ if successful otherwise $-1$ is returned along with an error
//! message to opserr. 





