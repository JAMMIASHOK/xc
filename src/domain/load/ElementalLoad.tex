%File: ~/OOP/element/ElementalLoad.tex
%What: "@(#) ElementalLoad.tex, revA"

\noindent {\bf Files}
\indent \#include \f$<\tilde{ }\f$/element/ElementalLoad.h\f$>\f$

\noindent {\bf Class Declaration}
\indent class ElementalLoad: public Load

\noindent {\bf Class Hierarchy}
\indent TaggedObject
\indent MovableObject
\indent\indent DomainComponent
\indent\indent\indent Load
\indent\indent\indent\indent {\bf ElementalLoad}

\noindent {\bf Description}
\indent ElementalLoad is an abstract class, i.e. no instances of
//! ElementalLoad will exist. The ElementalLoad class provides the
//! interface that all ElementalLoad writers must provide when
//! introducing new ElementalLoad classes. 

\noindent {\bf Class Interface}
\indent\indent // Constructors
\indent\indent {\em ElementalLoad(int elementTag, int tag, int classTag);} 
\indent\indent {\em ElementalLoad(int classTag);}
\indent\indent // Destructor
\indent\indent {\em virtual~ \f$\tilde{}\f$ElementalLoad();}
\indent\indent // Public Methods
\indent\indent {\em virtual int getElementTag(void) const;}



\noindent {\bf Constructor}
\indent {\em ElementalLoad(int elementTag, int tag, int classTag);}
//! Provided to allow subclasses to construct an ElementalLoad object
//! associated with the Element whose unique identifier in the Domain will
//! be \p elementTag. The integers \p tag and and \p classTags
//! are passed to the Load constructor. 

\indent {\em ElementalLoad(int classTag);}
//! Provided so that a FEM\_ObjectBroker can construct an object. \f$0\f$ and
\p classTag are passed to the Load classes constructor. The data
//! for the object is filled in when recvSelf() is invoked on the
//! object.

\noindent {\bf Destructor}
\indent {\em virtual~ \f$\tilde{}\f$ElementalLoad();} 
//! Does nothing. Provided so that the ElementalLoad subclasses destructor
//! will be called.

\noindent {\bf Public Methods }
\indent\indent {\em virtual int getElementTag(void) const;}
//! Returns the integer \p elementTag passed in the constructor. 
