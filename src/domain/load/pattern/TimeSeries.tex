%File: ~/OOP/domain/load/pattern/TimeSeries.tex
%What: "@(#) TimeSeries.tex, revA"

\noindent {\bf Files}   \\
\indent \#include $<\tilde{ }$domain/load/pattern/TimeSeries.h$>$  \\

\noindent {\bf Class Declaration}  \\
\indent class TimeSeries: public DomainComponent  \\

\noindent {\bf Class Hierarchy} \\
\indent MovableObject \\
\indent\indent {\bf TimeSeries} \\

\noindent {\bf Description} \\ 
\indent The TimeSeries class is an abstract base class. A
TimeSeries object is used in a LoadPattern to determine the current
load factor to be applied to the loads and constraints for the time
specified. \\ 

\noindent {\bf Class Interface} \\
\indent // Constructor \\ 
\indent {\em TimeSeries(int classTag);}\\ \\
\indent // Destructor \\ 
\indent {\em virtual $\tilde{ }$TimeSeries();}\\  \\
\indent // Pure Virtual Public Methods \\ 
\indent {\em  virtual double getFactor(double pseudoTime) =0;}\\
\indent {\em  virtual void Print(std::ostream \&s, int flag =0) =0;}\\

\noindent {\bf Constructor} \\ 
\indent {\em TimeSeries(int tag);}\\ 
The integer \p classTag is passed to the MovableObject classes
constructor. \\

\noindent {\bf Destructor} \\
\indent {\em virtual $\tilde{ }$TimeSeries();}\\ 
Does nothing. \\

\noindent {\bf Public Methods} \\
\indent {\em  virtual double getFactor(double pseudoTime) =0;}\\
To return the current load factor for the given value of {\em
pseudoTime} to be applied to the loads and single-point constraints in
a LoadPattern based on the value of \p pseudoTime. \\

\indent {\em  virtual void Print(std::ostream \&s, int flag =0) =0;}\\
To print to the stream \p s output based on the value of \p flag.
