% File: ~/domain/loadcase/Load.tex 

\noindent {\bf Files}
\indent \#include \f$<\tilde{ }\f$/domain/load/Load.h\f$>\f$

\noindent {\bf Class Declaration}
\indent class Load: public DomainComponent

\noindent {\bf Class Hierarchy}
\indent TaggedObject
\indent MovableObject
\indent\indent DomainComponent
\indent\indent\indent {\bf Load}

\noindent {\bf Description}
\indent Load is an abstract base class. A Load object is used to add
//! load to the domain. The Load class defines one method in its interface
//! applyLoad(), a method all subclasses must implement.


\noindent {\bf Class Interface}
\indent // Constructor
\indent {\em Load(tag, int classTag);}
\indent // Destructor
\indent {\em virtual \f$\tilde{ }\f$ Load();}
\indent // Public Methods
\indent {\em virtual void applyLoad(loadFactor) = 0;}
\indent {\em virtual void setLoadPatternTag(int loadPaternTag);}
\indent {\em virtual int  getLoadPatternTag(void) const;}

\noindent {\bf Constructor}
\indent {\em Load(tag, int classTag);}
//! Constructs a load with a tag given by \p tag and a class tag is
//! given by \p classTag. These are passed to the DomainComponent constructor.

\noindent {\bf Destructor}
\indent {\em virtual~\f$\tilde{}\f$ Load();}

\noindent {\bf Public Methods }
\indent {\em virtual void applyLoad(double loadFactor) = 0;}
//! The load object is to add \p loadFactor times the load to the
//! corresponding residual value at its associated element(s) or node(s).

\indent {\em virtual void setLoadPatternTag(int loadPaternTag);}
//! To set the tag of the enclosing load pattern for the load to be 
\p loadPatternTag.

\indent {\em virtual int  getLoadPatternTag(void) const;}
//! To return the current load pattern tag associated with the load. If no
//! load pattern tag has been set \f$-1\f$ is returned.

