%File: ~/OOP/element/node/NodalLoad.tex
%What: "@(#) NodalLoad.tex, revA"

\noindent {\bf Files}   \\
\indent \#include $<\tilde{ }$/domain/node/NodalLoad.h$>$  \\

\noindent {\bf Class Declaration}  \\
\indent class NodalLoad: public Load  \\

\noindent {\bf Class Hierarchy} \\
\indent TaggedObject \\
\indent MovableObject \\
\indent\indent DomainComponent \\
\indent\indent\indent Load \\
\indent\indent\indent\indent {\bf NodalLoad} \\

\noindent {\bf Description}  \\
\indent NodalLoads are loads acting on Nodes. The public methods are
all declared as virtual to allow subclasses to be introduced for the
provision of time varying loads. Each NodalLoad object is associated
with a single Node object and has a Vector object corresponding to the
load acting on this Node object as a result of the NodalLoad. \\

\noindent {\bf Class Interface}  \\
\indent\indent // Constructors  \\
\indent\indent {\em NodalLoad(int tag, int node, const Vector \&load, bool
isLoadConstant = false);}  \\ 
\indent\indent {\em NodalLoad(int tag, int node, int classTag);}  \\
\indent\indent {\em NodalLoad(int classTag);}  \\ \\
\indent\indent // Destructor  \\
\indent\indent {\em virtual $\tilde{ }$ NodalLoad();} \\ \\
\indent\indent // Public Methods  \\
\indent\indent {\em virtual void setDomain(Domain *newDomain);} \\
\indent\indent {\em virtual int getNodeTag(void) const;} \\
\indent\indent {\em virtual void applyLoad(double loadFactor);} \\ \\
\indent\indent // Public Methods for Output \\
\indent\indent {\em virtual int sendSelf(int commitTag, Channel \&theChannel); } \\
\indent\indent {\em virtual int recvSelf(int commitTag, Channel \&theChannel,
FEM\_ObjectBroker \&theBroker); } \\
\indent\indent {\em virtual void Print(std::stream \&s, int flag = 0);} \\


\noindent {\bf Constructors}  \\
\indent {\em NodalLoad(int tag, int node, const Vector \&load, bool
isLoadConstant = false);}  \\
To create a NodalLoad object with tag {\em tag} acting on Node {\em node} with a
reference load given by {\em load}. {\em tag} and {\em LOAD\_TAG\_NodalLoad} (defined in
$<$classTags.h$>$)are passed to the Load constructor.  A new vector object is created using
the vector {\em load} as the argument. (THIS MAY CHANGE - may associate
the load Vector with {\em load}, allowing identical loads on diff
nodes to be created without the need for each to have its own
vector) If no enough memory is available an error message is printed
and the program terminates. The boolean {\em isLoadConstant} is used
to indicate whether the value of the load applies to the Node is
independent of the load factor. \\ 

\indent {\em NodalLoad(int tag, int node, int classTag);}  \\
Provided for subclasses, which may which to provide the abstraction of
time varying nodal loads. The integers {\em tag} and {\em classTag}
are passed to the Load constructor. \\

\indent {\em NodalLoad(int classTag);}  \\
Provided so that a FEM\_ObjectBroker can construct an object. $0$ and
{\em classTag} are passed to the Load classes constructor. The data
for the object is filled in when {\em recvSelf()} is invoked on the
object.\\

\noindent {\bf Destructor}  \\
\indent {\em virtual $\tilde{ }$ NodalLoad();} \\ 
If a Vector for the load was constructed, the destructor invokes
delete on this Vector object. \\


\noindent {\bf Public Methods}  \\
\indent {\em virtual void setDomain(Domain *newDomain);} \\
To set the associated Domain object. First checks to ensure that a
Node object with the tag exists in the Domain. It sets the pointer for
it's associated Node object to point to this object, and then 
invokes the DomainComponent classes {\em setDomain()} method. If the
Node does not exist in the Domain a warning message and {\em
setDomain()} is not invoked. \\

\indent {\em virtual int getNodeTag(void) const;} \\
Returns the integer {\em node} passed in the constructor. \\

\indent {\em virtual void applyLoad(double timeStep = 0.0, double
loadFactor = 1.0);} \\
To it's associated Node it invokes {\em addUnbalancedLoad()} with it's
copy of the Vector object {\em load} and a factor of {\em loadFactor}
if {\em isLoadConstant} was specified as {\em false} in the constructor or
$1$ if it was specified as {\em true}. Warning
messages are printed, if no Domain has been associated with the
NodalLoad object or no Node with a tag {\em node} exists in the
Domain. For efficiency reasons, the NodalLoad object keeps a pointer
to it's associated Node object. The time {\em timeStep} has no
influence on the load added. \\

\indent {\em virtual int sendSelf(int commitTag, Channel \&theChannel); } \\
Determines its database tag. The object then sends it's tag, {\em
node} and size of load Vector to the Channel in an ID object. Then, if
{\em load} is not NULL, it sends it's copy of the {\em load}
Vector. Returns $0$ if successful, a negative number if the Channel
failed to send the data. \\  

\indent {\em virtual int recvSelf(int commitTag, Channel \&theChannel,
FEM\_ObjectBroker \&theBroker); } \\
The object first determines its database tag. It then invokes receives
an ID object from the Channel containing it's tag, {\em node} and size
of load Vector. If size of {\em load} is not $0$ it then receives it's
copy of the {\em load} Vector. Returns $0$ if successful, a negative 
number if the Channel failed to receive the data. \\ 

\indent {\em virtual void Print(std::stream \&s, int flag = 0);} \\
Prints it's {\em node} and then prints the load Vector. \\

