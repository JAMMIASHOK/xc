%File: ~/OOP/material/nD/NDMaterial.tex
%What: "@(#) NDMaterial.tex, revA"

//! INTERFACE MAY CHANGE IF MAKE MATERIAL MORE GENERAL.

\noindent {\bf Files}
\indent \#include \f$<\tilde{ }\f$/material/nD/NDMaterial.h\f$>\f$

\noindent {\bf Class Declaration}
\indent class NDMaterial : public Material

\noindent {\bf Class Hierarchy}
\indent TaggedObject
\indent MovableObject
\indent\indent Material
\indent\indent\indent {\bf NDMaterial}

\noindent {\bf Description}
\indent NDMaterial is an abstract class. The
//! NDMaterial class provides the interface that all
//! NDMaterial writers must provide when introducing new
//! NDMaterial subclasses. An NDMaterial object 
//! is responsible for keeping track of stress, strain and the
//! constitution for a particular point in the domain. 

\noindent {\bf Class Interface}
\indent // Constructor
\indent {\em NDMaterial (int tag, int classTag);}
\indent // Destructor
\indent {\em virtual \f$\tilde{ }\f$NDMaterial ();}
\indent // Public Methods
\indent {\em virtual int setTrialStrain (const Vector \&strain) = 0; }
\indent {\em virtual const Vector \&getStress (void) = 0; }
\indent {\em virtual const Matrix \&getTangent (void) = 0; }
\indent {\em virtual int commitState (void) = 0; }
\indent {\em virtual int revertToLastCommit (void) = 0; }
\indent {\em virtual int revertToStart (void) = 0; }
\indent {\em virtual NDMaterial *getCopy (void) = 0; }


\noindent {\bf Constructor}
\indent {\em NDMaterial (int tag, int classTag);}
//! To construct a NDMaterial whose unique integer among
//! NDMaterials in the domain is given by \p tag, and whose class
//! identifier is given by \p classTag. These integers are passed to
//! the Material class constructor.

\noindent {\bf Destructor}
\indent {\em virtual \f$\tilde{ }\f$NDMaterial ();}
//! Does nothing. 

\noindent {\bf Public Methods}
\indent {\em virtual int setTrialStrain (const Vector \&strain) = 0; }
//! Sets the value of the trial strain vector, that value used by {\em
//! getStress()} and getTangent(), to be \p strain. To return \f$0\f$
//! if successful and a negative number if not.

\indent {\em virtual const Vector \&getStress (void) = 0; }
//! To return the material stress vector at the current trial strain.

\indent {\em virtual const Matrix \&getTangent (void) = 0; }
//! To return the material tangent stiffness matrix at the current trial
//! strain.

\indent {\em virtual int commitState (void) = 0; }
//! To accept the current value of the trial strain vector as being on the
//! solution path. To return \f$0\f$ if successful, a negative number if not.

\indent {\em virtual int revertToLastCommit (void) = 0; }
//! To cause the material to revert to its last committed state. To
//! return \f$0\f$ if successful, a negative number if not.

\indent {\em virtual int revertToStart (void) = 0; }
//! Invoked to cause the material to revert to its original state in its
//! undeformed configuration. To return \f$0\f$ if successful, a negative
//! number if not.

\indent {\em virtual NDMaterial *getCopy (void) = 0; }
//! Returns a pointer to a new NDMaterial,
//! which is an exact copy of this instance. It is up to the caller to
//! ensure that the destructor is invoked.
