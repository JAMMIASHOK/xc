%File: ~/OOP/material/UniaxialMaterial.tex
%What: "@(#) UniaxialMaterial.tex, revA"

INTERFACE MAY CHANGE IF MAKE MATERIAL MORE GENERAL. \\

\noindent {\bf Files}   \\
\indent \#include $<\tilde{ }$/material/UniaxialMaterial.h$>$  \\

\noindent {\bf Class Declaration}  \\
\indent class UniaxialMaterial: public Material \\

\noindent {\bf Class Hierarchy} \\
\indent TaggedObject \\
\indent MovableObject \\
\indent\indent Material \\
\indent\indent\indent {\bf UniaxialMaterial} \\

\noindent {\bf Description}  \\
\indent UniaxialMaterial is an abstract class. The
UniaxialMaterial class provides the interface that all
UniaxialMaterial writers must provide when introducing new
UniaxialMaterial subclasses. A UniaxialMaterial object 
is responsible for keeping track of stress, strain and the
constitution for a particular point in the domain. \\ 

\noindent {\bf Class Interface} \\
\indent // Constructor \\
\indent {\em UniaxialMaterial (int tag, int classTag);}  \\ \\
\indent // Destructor \\
\indent {\em virtual $\tilde{ }$UniaxialMaterial ();}\\ \\
\indent // Public Methods \\
\indent {\em virtual int setTrialStrain (double strain) = 0; } \\
\indent {\em virtual double getStress (void) = 0; } \\
\indent {\em virtual double getTangent (void) = 0; } \\
\indent {\em virtual int commitState (void) = 0; } \\
\indent {\em virtual int revertToLastCommit (void) = 0; } \\
\indent {\em virtual int revertToStart (void) = 0; } \\
\indent {\em virtual UniaxialMaterial *getCopy (void) = 0; } \\


\noindent {\bf Constructor}  \\
\indent {\em UniaxialMaterial (int tag, int classTag);}  \\
To construct a UniaxialMaterial whose unique integer among
UniaxialMaterials in the domain is given by \p tag, and whose class
identifier is given by \p classTag. These integers are passed to
the Material class constructor. \\

\noindent {\bf Destructor} \\
\indent {\em virtual $\tilde{ }$UniaxialMaterial ();} \\
Does nothing. \\ 

\noindent {\bf Public Methods} \\
\indent {\em virtual int setTrialStrain (double strain) = 0; }  \\
Sets the value of the trial strain, that value used by {\em
getStress()} and getTangent(), to be \p strain. To
return $0$ if successful, a negative number if not. \\

\indent {\em virtual double getStress (void) = 0; } \\
To return the current value of stress for the trial strain. \\

\indent {\em virtual double getTangent (void) = 0; } \\
To return the current value of the tangent for the trial strain. \\

\indent {\em virtual int commitState (void) = 0; } \\
To accept the current value of the trial strain as being on the
solution path. To return $0$ if successful, a negative number if not. \\

\indent {\em virtual int revertToLastCommit (void) = 0; } \\
To cause the material to revert to the state at the last commit. To
return $0$ if successful, a negative number if not. \\

\indent {\em virtual int revertToStart (void) = 0; } \\
Invoked to cause the material to revert to its original state in its
undeformed configuration. To return $0$ if successful, a negative
number if not. \\

\indent {\em virtual UniaxialMaterial *getCopy (void) = 0; } \\
To return an exact copy of the material. \\
