%File: ~/OOP/system_of_eqn/linearSOE/profileSPD/ProfileSPDLinSubstrSolver.tex
%What: "@(#) ProfileSPDLinSubstrSolver.tex, revA"

//! UNDER CONSTRUCTION

\noindent {\bf Files}
\indent \#include \f$<\tilde{
}\f$/system\_of\_eqn/linearSOE/profileSPD/ProfileSPDLinSubstrSolver.h\f$>\f$ 

\noindent {\bf Class Decleration}
\indent class ProfileSPDLinSubstrSolver: public DomainSolver, public
//! ProfileSPDLinSubstrSolver 

\noindent {\bf Class Hierarchy}
\indent  MovableObject
\indent\indent  Solver
\indent\indent\indent  LinearSOESolver
\indent\indent\indent\indent  DomainSolver
\indent\indent\indent\indent  ProfileSPDLinDirectSolver
\indent\indent\indent\indent\indent {\bf ProfileSPDLinSubstrSolver}

\noindent {\bf Description}
\indent A ProfileSPDLinSubstrSolver object will perform the numerical
//! substructuring operations on a ProfileSPDLinSOE object. EXPAND.

\noindent {\bf Interface}
\indent\indent // Constructor
\indent\indent {\em ProfileSPDLinSubstrSolver(double tol = 1.0e-12);}
\indent\indent // Destructor
\indent\indent {\em \f$\tilde{ }\f$ProfileSPDLinSubstrSolver();}
\indent\indent // Public Methods
\indent\indent {\em int condenseA(int numInt);}
\indent\indent {\em int condenseRHS(int numInt, Vector *u = 0);}
\indent\indent {\em int computeCondensedMatVect(int numInt, Vector \&u);}
\indent\indent {\em Matrix \&getCondensedA(void);}
\indent\indent {\em Vector \&getCondensedRHS(void);}
\indent\indent {\em Vector \&getCondensedMatVect(void);}
\indent\indent {\em int setComputedXext(const Vector \&u);}
\indent\indent {\em  int solveXint(void);}






\noindent {\bf Constructor}
\indent {\em ProfileSPDLinSubstrSolver(double tol = 1.0e-12);}

\noindent {\bf Destructor}
\indent {\em \f$\tilde{ }\f$ProfileSPDLinSubstrSolver();} 

\noindent {\bf Public Methods}
\indent {\em int condenseA(int numInt);}
//! Causes the condenser to form \f$A_{ee}^* = A_{ee} -A_{ei} A_{ii}^{-1} A_{ie}\f$, where
\f$A_{ii}\f$ is the first \p numInt rows of the \f$A\f$ matrix.  The
//! original \f$A\f$ is changed as a result. \f$A_{ee}^*\f$ is to be stored in \f$A_{ee}\f$.

{\em int condenseRHS(int numInt);}
//! Causes the condenser to form \f$B_e^* = B_e - A_{ei} A_{ii}^{-1} B_i\f$, where \f$A_{ii}\f$ 
//! is the first \p numInt rows of \f$A\f$. The original \f$B\f$ is changed as a result. 
\f$B_e^*\f$ is to be stored in \f$B_e\f$.

{\em int computeCondensedMatVect(Vector \&u, int numInt);}
//! Causes the condenser to form \f$A_{ee} u\f$.

{\em Matrix \&getCondensedA(void);}
//! Returns the contents of \f$A_{ee}\f$ as a matrix.

{\em Vector \&getCondensedRHS(void);}
//! Returns the contents of \f$B_e\f$ as a Vector.

{\em Vector \&getCondensedMatVect(void);}
//! Returns the contents of the last call to {\em
//! computeCondensedMatVect()}.

{\em int setComputedXext(const Vector \&u);}
//! Sets the computed value of the unknowns in \f$X_e\f$ corresponding to the
//! external equations to \p u. The number of external equations is
//! given by the size of vector \f$u\f$.

{\em  int solveXint(void);}
//! To compute the internal equation numbers \f$X_i\f$ given the value set
//! for the external equations in the last call to setComputedXext().


