%File: ~/OOP/system_of_eqn/linearSOE/SparseGenCol/SuperLU.tex
%What: "@(#) SuperLU.tex, revA"

\noindent {\bf Files}
\indent \#include \f$<\tilde{
}\f$/system\_of\_eqn/linearSOE/fullGEN/SuperLU.h\f$>\f$ 

\noindent {\bf Class Decleration}
\indent class SuperLU: public SparseGenColLinSolver

\noindent {\bf Class Hierarchy}
\indent  MovableObject
\indent\indent  Solver
\indent\indent\indent LinearSOESolver
\indent\indent\indent\indent  SparseGenColLinSolver
\indent\indent\indent\indent\indent {\bf SuperLU}

\noindent {\bf Description}
\indent A SuperLU object can be constructed to solve
//! a SparseGenColLinSOE object. It obtains the solution by making calls on the
//! the SuperLU library developed at UC Berkeley by Prof. James Demmel, 
//! Xiaoye S. Li and John R. Gilbert.
//! The SuperLU library contains a set of subroutines to solve a sparse
//! linear system  \f$AX=B\f$. It uses Gaussian elimination with partial
//! pivoting (GEPP). The columns of A may be preordered before
//! factorization; the preordering for sparsity is completely separate
//! from the factorization and a number of ordering schemes are provided.

\noindent {\bf Interface}
\indent // Constructor
\indent {\em SuperLU();}
\indent // Destructor
\indent {\em \f$\tilde{ }\f$SuperLU();}
\indent // Public Methods
\indent {\em int solve(void);}
\indent {\em int setSize(void);}
\indent {\em int sendSelf(int commitTag, Channel \&theChannel);} 
\indent {\em int recvSelf(int commitTag, Channel \&theChannel,
//! FEM\_ObjectBroker \&theBroker);} 

\noindent {\bf Constructor}
\indent {\em SuperLU(int permSpec =0, double thresh = 0.0, int panelSize =6,
//! int relax = 6);}
//! A unique class tag (defined in \f$<\f$classTags.h\f$>\f$) is passed to the
//! SparseGenColLinSolver constructor. Saves the values for the arguments
\p permSpec, \p panelSize, \p relax and \p thresh that
//! will be used when calling the SuperLU routines in solve() and
//! setSize().

\p permSpec defines the ordering routine used in defining the
//! column permutations \p permC: \f$0\f$ uses the original ordering
//! supplied, \f$1\f$ defines a min-degree ordering based on \f$A^TA\f$ and \f$2\f$ a
//! min-degree ordering based on \f$A^T + A\f$. \p relax defines the min
//! number of columns in a subtree for the subtree to be considered a
//! single supernode. \p thresh defines the pivoting threshhold: at
//! step j of the Gaussian elimination if (abs\f$(A_{jj}) \ge\f$ \p thresh
(max\f$ i \ge j\f$ abs(\f$A_{ij}\f$)). A value for \p thresh of \f$0.0\f$
//! definines no pivoting, a value of \f$1.0\f$ classical partial pivoting.
\p panelSize defines the number of consequtive columns used as a
//! panel in the elimination. For more information on these values see the
//! SuperLU manual.


\noindent {\bf Destructor}
\indent {\em  \f$\tilde{ }\f$SuperLU();} 
//! Invokes delete on \p permR, \p permC and \p etree arrays.


\noindent {\bf Public Methods }
\indent {\em int solve(void);}
//! First copies \f$B\f$ into \f$X\f$ and then solves the FullGenLinSOE system 
//! it is associated with (pointer kept by parent class) by calling the SeuperLU
//! routine dgstrf(), if the system is marked as not having been factored,
//! or dgstrs(), if system is marked as having been factored. If the
//! solution is sucessfully obtained, i.e. the SuperLU routines return \f$0\f$
//! in the INFO argument, it marks the system has having been
//! factored and returns \f$0\f$, otherwise it prints a warning message and
//! returns INFO. The solve process changes \f$A\f$ and \f$X\f$ and sets the char
\p rafact to \p Y.   

\indent {\em int setSize(void);}
//! Obtains the size of the system from it's associaed SparseGenColLinSOE
//! object. With this information it creates space for the integer arrays
\p permR, \p permC and \p etree. It then creates the
//! a SuperMatrix for A by calling the SuperLU routine {\em
//! dCreate\_CompCol\_Matrix()}, sets the column permutation \p permR
//! by calling the SuperLU routine {\em get\_perm\_c(permSpec, A, permC)},
//! applies this permutation and determines the elimination tree {\em
//! etree} by calling the SuperLU routine sp\_preorder(). It then
//! creates a SuperMatrix for X by calling the SuperLU routine 
//! dCreate\_Dense\_Matrix().
//! Returns \f$0\f$ if sucessfull, prints a warning message and returns
//! a \f$-1\f$ if not enough memory is available for the arrays.


\indent {\em  int sendSelf(int commitTag, Channel \&theChannel);} 
//! Does nothing but return \f$0\f$.

\indent {\em  int recvSelf(int commitTag, Channel \&theChannel, FEM\_ObjectBroker
\&theBroker);} 
//! Does nothing but return \f$0\f$.








