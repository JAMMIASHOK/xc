%File: ~/OOP/system_of_eqn/linearSOE/bandSPD/BandSPDLinLapackSolver.tex
%What: "@(#) BandSPDLinLapackSolver.tex, revA"

\noindent {\bf Files}
\indent \#include \f$<\tilde{ }\f$/system\_of\_eqn/linearSOE/bandSPD/BandSPDLinLapackSolver.h\f$>\f$

\noindent {\bf Class Declaration}
\indent class BandSPDLinLapackSolver: public BandSPDLinSolver

\noindent {\bf Class Hierarchy}
\indent MovableObject
\indent\indent  Solver
\indent\indent\indent LinearSOESolver
\indent\indent\indent\indent BandSPDLinSolver
\indent\indent\indent\indent\indent {\bf BandSPDLinLapackSolver}

\noindent {\bf Description}
\indent A BandSPDLinLapackSolver object can be constructed to solve
//! a BandSPDLinSOE object. It obtains the solution by making calls on the
//! the LAPACK library. The class is defined to be a friend of the 
//! BandSPDLinSOE class (see \f$<\f$BandSPDLinSOE.h\f$>\f$).


\noindent {\bf Interface}
\indent\indent // Constructor
\indent\indent {\em BandSPDLinLapackSolver();}
\indent\indent // Destructor
\indent\indent {\em \f$\tilde{ }\f$BandSPDLinLapackSolver();}
\indent\indent // Public Methods
\indent\indent {\em int solve(void);}
\indent\indent {\em int setSize(void);}
\indent\indent {\em int sendSelf(int commitTag, Channel \&theChannel);} 
\indent\indent {\em int recvSelf(int commitTag, Channel \&theChannel,
//! FEM\_ObjectBroker \&theBroker);} 


\noindent {\bf Constructor}
\indent {\em BandSPDLinLapackSolver();}
//! A unique class tag (defined in \f$<\f$classTags.h\f$>\f$) is passed to the
//! BandSPDLinSolver constructor.


\noindent {\bf Destructor}
\indent {\em  \f$\tilde{ }\f$BandSPDLinLapackSolver();} 
//! Does nothing.

\noindent {\bf Public Member Functions }
\indent {\em virtual int solve(void);}
//! The solver first copies the B vector into X and then solves the
//! BandSPDLinSOE system by calling the LAPACK routines {\em 
//! dpbsv()}, if the system is marked as not having been factored,
//! and dpbtrs() if system is marked as having been factored. 
//! If the solution is successfully obtained, i.e. the LAPACK routines
//! return \f$0\f$ in the INFO argument, it marks the system has having been 
//! factored and returns \f$0\f$, otherwise it prints a warning message and
//! returns INFO. The solve process changes \f$A\f$ and \f$X\f$.   


\indent {\em int setSize(void);}
//! Does nothing but return \f$0\f$.

\indent {\em  int sendSelf(int commitTag, Channel \&theChannel);} 
//! Does nothing but return \f$0\f$.

\indent {\em  int recvSelf(int commitTag, Channel \&theChannel, FEM\_ObjectBroker
\&theBroker);} 
//! Does nothing but return \f$0\f$.



