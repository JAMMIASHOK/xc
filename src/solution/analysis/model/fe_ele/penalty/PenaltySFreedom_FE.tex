% File: ~/OOP/analysis/model/fe_ele/penalty/PenaltySFreedom_FE.tex 
%What: "@(#) PenaltySFreedom_FE.tex, revA"

\noindent {\bf Files}
\indent \#include \f$<\tilde{ }\f$/analysis/fe\_ele/penalty/PenaltySFreedom\_FE.h\f$>\f$

\noindent {\bf Class Declaration}
\indent class PenaltySFreedom\_FE: public FE\_Element ;

\noindent {\bf Class Hierarchy}
\indent FE\_Element
\indent\indent {\bf PenaltySFreedom\_FE} 

\noindent {\bf Description}
\indent PenaltySFreedom\_FE is a subclass of FE\_Element used to enforce a
//! single point constraint. It does this by adding \f$\alpha\f$ to the
//! tangent and \f$\alpha * (U\_s - U\_t)\f$ to the residual at the locations
//! corresponding to the constrained degree-of-freedom specified by the
//! SFreedom\_Constraint, where \f$U_s\f$ is the specified value of the constraint
//! and \f$U_t\f$ the current trial displacement at the node corresponding to
//! the constraint.


\noindent {\bf Class Interface}
\indent\indent // Constructor
\indent\indent {\em PenaltySFreedom\_FE(Domain \&theDomain, SFreedom\_Constraint
\&theSP, double alpha = 1.0e8);}
\indent\indent // Destructor
\indent\indent {\em virtual~ \f$\tilde{}\f$PenaltySFreedom\_FE();}
\indent\indent // Public Methods
\indent\indent {\em virtual void setID(void);} 
\indent\indent {\em virtual const Matrix \&getTangent(Integrator
*theIntegrator);}  
\indent\indent {\em virtual const Vector \&getResidual(Integrator
*theIntegrator);} 
\indent\indent {\em virtual const Vector \&getTangForce(const Vector
\&disp, double fact = 1.0);    }

\noindent {\bf Constructor}
\indent {\em PenaltySFreedom\_FE(Domain \&theDomain, SFreedom\_Constraint \&theSP,
//! double alpha = 1.0e8);}
//! To construct a PenaltySFreedom\_FE element to enforce the constraint
//! specified by the SFreedom\_Constraint \p theSP using a value for
\f$\alpha\f$ of \p alpha (which, if none is specified, defaults to
\f$1.0e8\f$). The FE\_Element class constructor is called with 
//! the integers \f$1\f$ and \f$1\f$. A Matrix and a Vector object of order \f$1\f$
//! are created to return the tangent and residual contributions, with the
//! tangent entry being set at \f$\alpha\f$. A link to the Node in the  Domain
//! corresponding to the SFreedom\_Constraint is also set. A warning message is
//! printed and program terminates if there is not enough memory or no
//! Node associated with the SFreedom\_Constraint exists in the Domain.

\noindent {\bf Destructor}
\indent {\em virtual~ \f$\tilde{}\f$PenaltySFreedom\_FE();}
//! Invokes the destructor on the Matrix and Vector objects created in the
//! constructor.

\noindent {\bf Public Methods}
\indent {\em virtual void setID(void);}
//! Causes the PenaltySFreedom\_FE to determine the mapping between it's equation
//! numbers and the degrees-of-freedom. From the Node object link, created
//! in the constructor, the DOF\_group corresponding to the Node
//! associated with the constraint is determined. From this {\em
//! DOF\_Group} object the mapping for the constrained degree of freedom
//! is determined and the ID in the base class is set. Returns \f$0\f$ if
//! successful. Prints a warning message and returns a negative number if
//! an error occurs: \f$-2\f$ if the
//! Node has no associated DOF\_Group, \f$-3\f$ if the constrained DOF
//! specified is invalid for this Node and \f$-4\f$ if the ID in the
//! DOF\_Group is too small for the Node. 

\indent {\em virtual const Matrix \&getTangent(Integrator *theIntegrator);}
//! Returns the tangent Matrix created in the constructor.

\indent {\em virtual const Vector \&getResidual(Integrator *theIntegrator);}
//! Sets the FE\_Elements contribution to the residual to be
\f$\alpha * (U_s - U_t)\f$, where \f$U_s\f$ is the specified value of the
//! constraint and \f$U_t\f$ the current trial displacement at the node
//! corresponding to constrained degree-of-freedom. Prints a warning
//! message and sets this contribution to \f$0\f$ if the specified constrained
//! degree-of-freedom is invalid. Returns this residual Vector set.


{\em virtual const Vector \&getTangForce(const Vector \&disp, double
//! fact = 1.0);    }
//! Sets the FE\_Elements contribution to the residual to be
\f$\alpha * (U\_s - disp\_t)\f$, where \f$U\_s\f$ is the specified value of the
//! constraint and \f$disp\_t\f$ the value in \p disp
//! corresponding to constrained degree-of-freedom. Prints a warning
//! message and sets this contribution to \f$0\f$ if the mapping, determined in
//! setID(), for the the specified constrained degree-of-freedom lies 
//! outside \p disp.  










