% File: ~/OOP/analysis/model/fe_ele/penalty/PenaltyMFreedom_FE.tex 
%What: "@(#) PenaltyMFreedom_FE.tex, revA"

\noindent {\bf Files}
\indent \#include \f$<\tilde{ }\f$/analysis/fe\_ele/penalty/PenaltyMP\_FE.h\f$>\f$

\noindent {\bf Class Declaration}
\indent class PenaltyMP\_FE: public FE\_Element ;

\noindent {\bf Class Hierarchy}
\indent FE\_Element
\indent\indent {\bf PenaltyMP\_FE} 

\noindent {\bf Description}
\indent PenaltyMP\_FE is a subclass of FE\_Element used to enforce a
//! multi point constraint, of the form \f$\U_c = \C_{cr} \U_r\f$, where \f$\U_c\f$ are
//! the constrained degrees-of-freedom at the constrained node, \f$\U_r\f$ are
//! the retained degrees-of-freedom at the retained node and \f$\C_{cr}\f$ a
//! matrix defining the relationship between these degrees-of-freedom. 

//! To enforce the constraint a matrix \f$\alpha \C^T \C\f$ is added to the
//! tangent for the degrees-of-freedom \f$[\U_c\f$ \f$\U_r]\f$, where \f$\C = [-\I\f$ 
\f$\C_{cr}]\f$. Nothing is added to the residual.  

\noindent {\bf Class Interface}
\indent\indent // Constructor
\indent\indent {\em PenaltyMP\_FE(Domain \&theDomain, MP\_Constraint
\&theMFreedom, double alpha);}
\indent\indent // Destructor
\indent\indent {\em virtual~ \f$\tilde{}\f$PenaltyMP\_FE();}
\indent\indent // Public Methods
\indent\indent {\em virtual void setID(void);} 
\indent\indent {\em virtual const Matrix \&getTangent(Integrator
*theIntegrator);}  
\indent\indent {\em virtual const Vector \&getResidual(Integrator
*theIntegrator);} 
\indent\indent {\em virtual const Vector \&getTangForce(const Vector
\&disp, double fact = 1.0);    }

\noindent {\bf Constructor}
\indent {\em PenaltyMP\_FE(Domain \&theDomain, MP\_Constraint \&theMFreedom,
//! double alpha);}
//! To construct a PenaltyMP\_FE element to enforce the constraint
//! specified by the MP\_Constraint \p theMFreedom using a default value for
\f$\alpha\f$ of \f$alpha\f$. The FE\_Element class constructor is called with
//! the integers \f$2\f$ and the size of the \p retainedID plus the size of
//! the \p constrainedID at the MP\_Constraint \p theMFreedom. A Matrix
//! and a Vector object are created for adding the contributions to the
//! tangent and the residual. The residual is zeroed. A Matrix is created
//! to store the \f$C\f$ Matrix. If the MP\_Constraint is not time varying,
//! the components of this Matrix are determined, then the contribution
//! to the tangent \f$\alpha C^TC\f$ is determined and finally the \f$C\f$ matrix
//! is destroyed. Links are set to the retained and constrained nodes.
//! A warning message is printed and the program is terminated if
//! either not enough memory is available for the Matrices and Vector or the
//! constrained and retained Nodes do not exist in the Domain.


\noindent {\bf Destructor}
\indent {\em virtual~ \f$\tilde{}\f$PenaltyMP\_FE();}
//! Invokes delete on any Matrix or Vector objects created in the
//! constructor that have not yet been destroyed.

\noindent {\bf Public Methods}
\indent {\em virtual void setID(void);}
//! Causes the PenaltyMP\_FE to determine the mapping between it's equation
//! numbers and the degrees-of-freedom. This information is obtained by
//! using the mapping information at the DOF\_Group objects associated with
//! the constrained and retained nodes to determine the mappings between
//! the degrees-of-freedom identified in the \p constrainedID and the
\p retainedID at the MP\_Constraint \p theMFreedom. Returns \f$0\f$ if
//! successful. Prints a warning message and returns a negative number if
//! an error occurs: \f$-2\f$ if the
//! Node has no associated DOF\_Group, \f$-3\f$ if the constrained DOF
//! specified is invalid for this Node (sets corresponding ID component to
\f$-1\f$ so nothing is added to the tangent) and \f$-4\f$ if the ID in the
//! DOF\_Group is too small for the Node (again setting corresponding ID
//! component to \f$-1\f$). 


\indent {\em virtual Matrix \&getTangent(Integrator *theIntegrator);}
//! If the MP\_Constraint is time-varying, from the MP\_Constraint
\p theMFreedom it obtains the current \f$C_{cr}\f$ matrix; it then forms the
\f$C\f$ matrix and finally it sets the tangent matrix to be \f$\alpha
//! C^TC\f$. Returns the tangent matrix.

\indent {\em virtual const Vector \&getResidual(Integrator *theIntegrator);}
//! Returns the residual, a \f$\zero\f$ Vector.

{\em virtual const Vector \&getTangForce(const Vector \&disp, double
//! fact = 1.0);    }
//! CURRENTLY just returns the \f$0\f$ residual. THIS WILL NEED TO CHANGE FOR
//! ELE-BY-ELE SOLVERS. 

