%File: ~/OOP/analysis/integrator/HHT.tex
%What: "@(#) HHT.tex, revA"

\noindent {\bf Files}
\indent \#include \f$<\tilde{ }\f$/analysis/integrator/HHT.h\f$>\f$

\noindent {\bf Class Declaration}
\indent class HHT: public TransientIntegrator

\noindent {\bf Class Hierarchy}
\indent MovableObject
\indent\indent Integrator
\indent\indent\indent IncrementalIntegrator
\indent\indent\indent\indent TransientIntegrator
\indent\indent\indent\indent\indent {\bf HHT}

\noindent {\bf Description} 
\indent HHT is a subclass of TransientIntegrator which implements
//! the Hilber-Hughes-Taylor (HHT) method. In the HHT method, to determine the
//! velocities, accelerations and displacements at time \f$t + \Delta t\f$,
//! by solving the following equilibrium equation

\[ \R (\U_{t + \Delta t}) = \P(t + \Delta t) -
\F_I(\Udd_{t+\Delta t}) - \F_R(\Ud_{t + \alpha \Delta t},\U_{t +
\alpha \Delta t}) \] 

\noindent where

\[ \U_{t + \alpha} = \left( 1 - \alpha \right) \U_t + \alpha \U_{t +
\Delta t} \]

\[ \Ud_{t + \alpha} = \left( 1 - \alpha \right) \Ud_t + \alpha \Ud_{t +
\Delta t} \]

\noindent and the velocities and accelerations at time \f$t + \Delta t\f$
//! are determined using the Newmark relations. The HHT method results in
//! the following for determining the response at \f$t + \Delta t\f$

\[ \left[ \frac{1}{\beta \Delta t^2} \M + \frac{\alpha \gamma}{\beta
\Delta t} \C + \alpha \K \right] \Delta \U_{t + \Delta t}^{(i)} = \P(t
+ \Delta t) - \F_I\left(\Udd_{t+\Delta  t}^{(i-1)}\right)
- \F_R\left(\Ud_{t + \alpha \Delta t}^{(i-1)},\U_{t + \alpha \Delta
//! t}^{(i-1)}\right) \] 


\noindent {\bf Class Interface}
\indent // Constructors
\indent {\em HHT();} 
\indent {\em HHT(double alpha);} 
\indent {\em HHT(double alpha, double alphaM, double betaK);}
\indent // Destructor
\indent {\em virtual~ \f$\tilde{}\f$HHT();}
\indent // Public Methods
\indent {\em int formEleTangent(FE\_Element *theEle);}
\indent {\em int formNodTangent(DOF\_Group *theDof);}
\indent {\em int domainChanged(void);}
\indent {\em int newStep(double deltaT);}
\indent {\em int update(const Vector \&\f$\Delta U\f$);} 
\indent {\em int commit(void);}
\indent // Public Methods for Output
\indent {\em int sendSelf(int commitTag, Channel \&theChannel);} 
\indent {\em int recvSelf(int commitTag, Channel \&theChannel,
//! FEM\_ObjectBroker \&theBroker);} 
\indent {\em int Print(OPS\_Stream \&s, int flag = 0);}

\noindent {\bf Constructors}
\indent {\em HHT();} 
//! The integer INTEGRATOR\_TAGS\_HHT is passed to the TransientIntegrator
//! constructor. \f$\alpha\f$, \f$\beta\f$ and \f$\gamma\f$ are set to 0.0. This
//! constructor should only be invoked by an FEM\_ObjectBroker.

\indent {\em HHT(double alpha);} 
//! Sets \f$\alpha\f$ to \p alpha, \f$\gamma\f$ to \f$(1.5 - \alpha)\f$ and \f$\beta\f$
//! to \f$0.25*\alpha^2\f$. In addition, a flag is set indicating that Rayleigh
//! damping will not be used.

\indent {\em HHT(double alpha, double alphaM, double betaK);} 
//! This constructor is invoked if Rayleigh damping is to be used, 
//! i.e. \f$\D = \alpha_M M + \beta_K K\f$. Sets \f$\alpha\f$ to \p alpha,
\f$\gamma\f$ to \f$(1.5 - \alpha)\f$, \f$\beta\f$ to \f$0.25*\alpha^2\f$, \f$\alpha_M\f$ to
\p alphaM and \f$\beta_K\f$ to \p betaK. Sets a flag indicating
//! whether the incremental solution is done in terms of displacement or
//! acceleration to \p dispFlag and a flag indicating that Rayleigh
//! damping will be used. 

//! Sets \f$\alpha\f$ to \p alpha, \f$\gamma\f$ to \f$(1.5 - \alpha)\f$ and \f$\beta\f$
//! to \f$0.25*\alpha^2\f$. In addition, a flag is set indicating that Rayleigh
//! damping will not be used.

\noindent {\bf Destructor}
\indent {\em virtual~ \f$\tilde{}\f$HHT();} 
//! Invokes the destructor on the Vector objects created.

\noindent {\bf Public Methods}
\indent {\em int formEleTangent(FE\_Element *theEle);}
//! This tangent for each FE\_Element is defined to be \f$\K_e = c1\alpha \K
+ c2\alpha \D + c3 \M\f$, where c1,c2 and c3 were determined in the last invocation
//! of the newStep() method. Returns \f$0\f$ after performing the
//! following operations:  
\begin{tabbing}
//! while \= \+ while \= while \= \kill
//! if (RayleighDamping == false) \{ \+
//! theEle-\f$>\f$zeroTang()
//! theEle-\f$>\f$addKtoTang(c1)
//! theEle-\f$>\f$addCtoTang(c2)
//! theEle-\f$>\f$addMtoTang(c3)  \-
\} else \{ \+
//! theEle-\f$>\f$zeroTang()
//! theEle-\f$>\f$addKtoTang(c1 + c2 * \f$\beta_K\f$)
//! theEle-\f$>\f$addMtoTang(c3 + c2 * \f$\alpha_M\f$)  \- 
\}
\end{tabbing}

{\em int formNodTangent(DOF\_Group *theDof);}
//! This performs the following:
\begin{tabbing}
//! while \= \+ while \= while \= \kill
//! if (RayleighDamping == false)  \+
//! theDof-\f$>\f$addMtoTang(c3)  \-
//! else \+
//! theDof-\f$>\f$addMtoTang(c3 + c2 * \f$\alpha_M\f$)  \- 
\end{tabbing}


{\em int domainChanged(void);}
//! If the size of the LinearSOE has changed, the object deletes any old Vectors
//! created and then creates \f$8\f$ new Vector objects of size equal to {\em
//! theLinearSOE-\f$>\f$getNumEqn()}. There is a Vector object created to store
//! the current displacement, velocity and accelerations at times \f$t\f$ and
\f$t + \Delta t\f$, and the displacement and velocity at time \f$t + \alpha
\Delta t\f$. The response quantities at time \f$t + \Delta t\f$ are
//! then set by iterating over the DOF\_Group objects in the model and
//! obtaining their committed values. 
//! Returns \f$0\f$ if successful, otherwise a warning message and a negative
//! number is returned: \f$-1\f$ if no memory was available for constructing
//! the Vectors.


{\em int newStep(double \f$\Delta t\f$);}
//! The following are performed when this method is invoked:
\begin{enumerate}
\item First sets the values of the three constants {\em c1}, {\em c2}
//! and {\em c3}, {\em c1} is set to \f$1.0\f$, {\em c2} to \f$
\gamma / (\beta * \Delta t)\f$ and {\em c3} to \f$1/ (\beta * \Delta t^2)\f$.
\item Then the Vectors for response quantities at time \f$t\f$ are set
//! equal to those at time \f$t + \Delta t\f$.
\begin{tabbing}
//! while \= while \= while \= while \= \kill
\>\> \f$ \U_t = \U_{t + \Delta t}\f$
\>\> \f$ \Ud_t = \Ud_{t + \Delta t} \f$
\>\> \f$ \Udd_t = \Udd_{t + \Delta t} \f$ 
\end{tabbing}
\item Then the velocity and accelerations approximations at time \f$t +
\Delta t\f$ and the displacement and velocity at time \f$t + \alpha \Delta t\f$
//! are set using the difference approximations.
\begin{tabbing}
//! while \= while \= while \= while \= \kill
\>\> \f$ \U_{t + \alpha \Delta t} = \U_t\f$
\>\> \f$ \dot \U_{t + \Delta t} = 
 \left( 1 - \frac{\gamma}{\beta}\right) \dot \U_t + \Delta t \left(1
- \frac{\gamma}{2 \beta}\right) \ddot \U_t \f$
\>\> \f$ \Ud_{t + \alpha \Delta t} = (1 - \alpha) \Ud_t + \alpha \Ud_{t +
\Delta t}\f$
\>\> \f$ \ddot \U_{t + \Delta t} = 
 - \frac{1}{\beta \Delta t} \dot \U_t + \left( 1 - \frac{1}{2
\beta} \right) \ddot \U_t  \f$
\>\> theModel-\f$>\f$setResponse\f$(\U_{t + \alpha \Delta t}, \Ud_{t+\alpha
\Delta t}, \Udd_{t+\Delta t})\f$ 
\end{tabbing}
\item The response quantities at the DOF\_Group objects are updated
//! with the new approximations by invoking setResponse() on the
//! AnalysisModel with displacements and velocities at time \f$t + \alpha
\Delta t\f$ and the accelerations at time \f$t + \Delta t\f$.
\begin{tabbing}
//! while \= while \= while \= while \= \kill
\>\> theModel-\f$>\f$setResponse\f$(\U_{t + \alpha \Delta t}, \Ud_{t+\alpha
\Delta t}, \Udd_{t+\Delta t})\f$ 
\end{tabbing}
\item current time is obtained from the AnalysisModel, incremented by
\f$\Delta t\f$, and {\em applyLoad(time, 1.0)} is invoked on the
//! AnalysisModel. 
\item Finally updateDomain() is invoked on the AnalysisModel.
\end{enumerate}
//! The method returns \f$0\f$ if successful, otherwise a negative number is
//! returned: \f$-1\f$ if \f$\gamma\f$ or \f$\beta\f$ are \f$0\f$, \f$-2\f$ if \p dispFlag
//! was true and \f$\Delta t\f$ is \f$0\f$, and \f$-3\f$ if domainChanged()
//! failed or has not been called.



{\em int update(const Vector \&\f$\Delta U\f$);}
//! Invoked this first causes the object to increment the DOF\_Group
//! response quantities at time \f$t + \Delta t\f$. The displacement Vector is  
//! incremented by \f$ c1 * \Delta U\f$, the velocity Vector by \f$
//! c2 * \Delta U\f$, and the acceleration Vector by \f$c3 * \Delta U\f$. 
//! The displacement Vector at time \f$t + \alpha \Delta t\f$ is incremented
//! by \f$c1 \alpha \Delta U\f$ and the velocity Vector by \f$c2 \alpha \Delta U\f$.
//! The response quantities at the DOF\_Group objects are then updated
//! with the new approximations by invoking setResponse() on the
//! AnalysisModel with displacement and velocity at time \f$t + \alpha
\Delta t\f$ and the accelerations at time \f$t + \Delta t\f$. 
//! Finally updateDomain() is invoked on the AnalysisModel. 
\begin{tabbing}
//! while \= \+ while \= while \= \kill
\>\> \f$ \U_{t + \Delta t} += \Delta \U\f$
\>\> \f$ \dot \U_{t + \Delta t} += \frac{\gamma}{\beta \Delta t} \Delta \U \f$
\>\> \f$ \ddot \U_{t + \Delta t} += \frac{1}{\beta {\Delta t}^2} \Delta \U \f$
\>\> \f$ \U_{t + \alpha \Delta t} += \alpha \Delta \U \f$
\>\> \f$ \Ud_{t + \alpha \Delta t} += \frac{\alpha \gamma}{\beta \Delta t}
\Delta \U \f$ 
\>\> theModel-\f$>\f$setResponse\f$(\U_{t + \alpha \Delta t}, \Ud_{t+\alpha
\Delta t}, \Udd_{t+\Delta t})\f$
\>\> theModel-\f$>\f$updateDomain()
\end{tabbing}
//! Returns
\f$0\f$ if successful. A warning message is printed and a negative number
//! returned if an error occurs: \f$-1\f$ if no associated AnalysisModel, \f$-2\f$
//! if the Vector objects have not been created, \f$-3\f$ if the Vector
//! objects and \f$\Delta U\f$ are of different sizes.


{\em int commit(void);}
//! First the response quantities at the DOF\_Group objects are updated
//! with the new approximations by invoking setResponse() on the
//! AnalysisModel with displacement, velocity and accelerations at time \f$t +
\Delta t\f$. Finally updateDomain()} and {\em commitDomain() are
//! invoked on the AnalysisModel. 
//! Returns \f$0\f$ if successful, a warning
//! message and a negative number if not: \f$-1\f$ if no AnalysisModel
//! associated with the object and \f$-2\f$ if commitDomain() failed.


{\em int sendSelf(int commitTag, Channel \&theChannel); } 
//! Places in a Vector of size 6 the values of \f$\alpha\f$, \f$\beta\f$, \f$\gamma\f$, 
//! RayleighDampingFlag, \f$\alpha_M\f$ and \f$\beta_K\f$.  Then
//! invokes sendVector() on the Channel with this Vector. Returns
\f$0\f$ if successful, a warning message is printed and a \f$-1\f$ is
//! returned if \p theChannel fails to send the Vector. 

{\em int recvSelf(int commitTag, Channel \&theChannel, 
//! FEM\_ObjectBroker \&theBroker); } 
//! Receives in a Vector of size 6 the values of \f$\alpha\f$, \f$\beta\f$, \f$\gamma\f$, 
//! RayleighDampingFlag, \f$\alpha_M\f$ and \f$\beta_K\f$. Returns \f$0\f$
//! if successful. A warning message is printed, and a \f$-1\f$ is returned if
\p theChannel fails to receive the Vector.

{\em int Print(OPS\_Stream \&s, int flag = 0);}
//! The object sends to \f$s\f$ its type, the current time, \f$\alpha\f$, \f$\gamma\f$ and
\f$\beta\f$. If Rayleigh damping is specified, the constants \f$\alpha_M\f$ and
\f$\beta_K\f$ are also printed.


