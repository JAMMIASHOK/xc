%File: ~/OOP/analysis/integrator/LoadControl.tex
%What: "@(#) LoadControl.tex, revA"

\noindent {\bf Files}
\indent \#include \f$<\tilde{ }\f$/analysis/integrator/LoadControl.h\f$>\f$

\noindent {\bf Class Declaration}
\indent class LoadControl: public StaticIntegrator

\noindent {\bf Class Hierarchy}
\indent MovableObject
\indent\indent Integrator
\indent\indent\indent IncrementalIntegrator
\indent\indent\indent\indent StaticIntegrator
\indent\indent\indent\indent\indent {\bf LoadControl}

\noindent {\bf Description} 
\indent LoadControl is a subclass of StaticIntegrator, it is
//! used to when performing a static analysis on the FE\_Model using the
//! load control method. In the load control method, the following
//! constraint equation is added to equation~\ref{staticFormTaylor} of the
//! StaticIntegrator class: 

\[ 
\lambda_n^{(i)} - \lambda_{n-1} = \delta \lambda_n
\]

\noindent where \f$\delta \lambda_n\f$ depends on \f$\delta \lambda_{n-1}\f$,
//! the load increment at the previous time step, \f$J_{n-1}\f$,
//! the number of iterations required to achieve convergence in the
//! previos load step, and \f$Jd\f$, the desired number of iteraions. \f$\delta
\lambda_n\f$ is bounded by \f$\delta \lambda_{min}\f$  and \f$\delta \lambda_{max}\f$.


\[ 
\delta \lambda_n = max \left( \delta \lambda_{min}, min \left(
\frac{Jd}{J_{n-1}} \delta \lambda_{n-1}, \delta \lambda_{max} \right) \right)
\]

//! Knowing \f$\lambda_n^{(i)}\f$ prior to each iteration, the \f$N+1\f$ unknowns
//! in equation~\ref{staticFormTaylor}, is reduced to \f$N\f$ unknowns and
//! results in the following equation:

\begin{equation} 
\R(\U_{n}) = \lambda_n^{(i)} \P 
 - \f_{R}\left(\U_{n}^{(i)} \right) - 
\K_n^{(i)} 
(\U_{n} - \U_{n}^{(i)})  
\label{staticFormLoadControl}
\end{equation} 

\noindent {\bf Class Interface}
\indent // Constructors
\indent {\em LoadControl(double \f$\delta \lambda_1\f$, int Jd, 
//! double \f$\delta \lambda_{min}\f$, double \f$\delta \lambda_{max}\f$);}
\indent // Destructor
\indent {\em \f$\tilde{ }\f$LoadControl();}
\indent // Public Methods
\indent {\em int newStep(void);}
\indent {\em int update(const Vector \&\f$\Delta U\f$);}
\indent {\em int setDeltaLambda(double \f$\delta
\lambda\f$;)}
\indent // Public Methods for Output
\indent {\em int sendSelf(int commitTag, Channel \&theChannel);} 
\indent {\em int recvSelf(int commitTag, Channel \&theChannel,
//! FEM\_ObjectBroker \&theBroker);} 
\indent {\em int Print(OPS\_Stream \&s, int flag = 0);}

\noindent {\bf Constructors}
\indent {\em LoadControl(double \f$\delta \lambda_1\f$, int Jd, 
//! double \f$\delta \lambda_{min}\f$, double \f$\delta \lambda_{max}\f$);}
//! The integer INTEGRATOR\_TAGS\_LoadControl (defined in
\f$<\f$classTags.h\f$>\f$) is passed to the StaticIntegrator classes
//! constructor. \f$\delta \lambda_1\f$ is the load factor used in the first
//! step. The arguments \f$Jd\f$, \f$\delta \lambda_{min}\f$, and \f$\delta
\lambda_{max}\f$ are used in the determination of the increment in the
//! load factor at each step.



\noindent {\bf Destructor}
\indent {\em \f$\tilde{ }\f$LoadControl();} 
//! Does nothing.

\noindent {\bf Public Methods}

{\em int newStep(void);}
//! The object obtains the current value of \f$\lambda\f$ from the AnalysisModel
//! object. It increments this by \f$\delta \lambda_n \f$.

\[ 
\delta \lambda_n = max \left( \delta \lambda_{min}, min \left(
\frac{Jd}{J_{n-1}} \delta \lambda_{n-1}, \delta \lambda_{max} \right) \right)
\]

\noindent It will then invoke
{\em applyLoadDomain(0.0, \f$\lambda\f$)} on the AnalysisModel
//! object. Returns \f$0\f$ if successful. A warning message is printed and a
\f$-1\f$ is returned if no AnalysisModel is associated with the object.

{\em int update(const Vector \&\f$\Delta U\f$);}
//! Invoked this causes the object to first increment the DOF\_Group
//! displacements with \f$\Delta U\f$, by invoking {\em incrDisp(\f$\Delta U)\f$}
//! on the AnalysisModel, and then to update the domain, by invoking {\em
//! updateDomain()} on the AnalysisModel. Returns \f$0\f$ if successful, a
//! warning message and a \f$-1\f$ is returned if no AnalysisModel is
//! associated with the object.


\indent {\em int setDeltaLambda(double \f$\delta \lambda\f$;)}
//! Sets the value of the load increment in newStep() to be \f$\delta
\lambda\f$. Returns \f$0\f$.

{\em int sendSelf(int commitTag, Channel \&theChannel); } 
//! Places in a vector if size 5 the value of \f$\delta \lambda_{n-1}\f$,
\f$Jd\f$, \f$J_{n-1}\f$, \f$\delta \lambda_{min}\f$ and \f$\delta \lambda_{max}\f$)
//! and then sends the Vector. Returns \f$0\f$ if successful, a warning
//! message is printed and a \f$-1\f$ is returned if \p theChannel fails to
//! send the Vector. 

{\em int recvSelf(int commitTag, Channel \&theChannel, 
//! FEM\_ObjectBroker \&theBroker); } 
//! Receives in a Vector of size 5 the data that was sent in sendSelf().
//! Returns \f$0\f$ if successful, a warning message is printed, \f$\delta
\lambda\f$ is set to \f$0\f$, and a \f$-1\f$ is returned if \p theChannel 
//! fails to receive the Vector.

{\em int Print(OPS\_Stream \&s, int flag = 0);}
//! The object sends to \f$s\f$ its type, the current value of \f$\lambda\f$, and
\f$\delta \lambda\f$. 
