%File: ~/OOP/convergenceTest/CTestEnergyIncr.tex
%What: "@(#) CTestEnergyIncr.tex, revA"

\noindent {\bf Files}   \\
\indent \#include $<\tilde{}$/convergenceTest/CTestEnergyIncr.h$>$  \\

\noindent {\bf Class Declaration}  \\
\indent class CTestEnergyIncr: public ConvergenceTest  \\

\noindent {\bf Class Hierarchy} \\
\indent MovableObject \\
\indent\indent ConvergenceTest \\
\indent\indent\indent {\bf CTestEnergyIncr} \\

\noindent {\bf Description}  \\
\indent A CTestEnergyIncr object is an object which can be used in an
algorithmic class to test if convergence has been achieved. The
CTestEnergyIncr class tests using the energy increment, 0.5 times the
absolute value of the dot product of the LinearSOE objects solution
and rhs Vectors, and a user specified tolerance value. \\


\noindent {\bf Class Interface} \\
\indent\indent {// Constructors}  \\ 
\indent\indent {\em CTestEnergyIncr(double tol, int maxNumIter, double
printFlag);}  \\  
\indent\indent {\em CTestEnergyIncr();}  \\ \\
\indent\indent {// Destructor}  \\ 
\indent\indent {\em $\tilde{}$CTestEnergyIncr();} \\ \\
\indent\indent {// Public Methods}  \\ 
\indent\indent {\em int setTolerance(double newTol);} \\
\indent\indent {\em int setEquiSolnAlgo(EquiSolnAlgo \&theAlgo);} \\
\indent\indent {\em int start(void);} \\
\indent\indent {\em int test(void);} \\
\indent\indent {\em int sendSelf(int commitTag, Channel \&theChannel);}\\ 
\indent\indent {\em int recvSelf(int commitTag, Channel \&theChannel,
FEM\_ObjectBroker \&theBroker);}\\ 

\noindent {\bf Constructors}  \\
\indent {\em CTestEnergyIncr(double tol, int maxNumIncr = 10);}  \\
The integer {\em CLASS\_TAGS\_CTestEnergyIncr}, defined in
$<$classTags.h$>$, is passed to the ConvergenceTest constructor. Sets
the tolerance used in test() to be \p tol and the max number
of iterations to be performed to \p maxNumIter. \\

\indent {\em CTestEnergyIncr();}  \\
To be used by the FEM\_ObjectBroker object in parallel programs. The
integer {\em CLASS\_TAGS\_CTestEnergyIncr}, defined in
$<$classTags.h$>$, is passed to the ConvergenceTest constructor. Sets
the tolerance used in test() to be {\em 0.0} and \p maxNumIter 
to be $0$. These will be set when recvSelf() is invoked on the object. \\


\noindent {\bf Destructor} \\
\indent {\em $\tilde{ }$CTestEnergyIncry();} \\ 
Does nothing. \\

\noindent {\bf Public Methods }  \\
\indent {\em int setTolerance(double newTol);} \\
Sets the tolerance used in test() to be \p newTol. \\

\indent {\em int setEquiSolnAlgo(EquiSolnAlgo \&theAlgo);} \\
It sets a pointer to \p theAlgo's LinearSOE object. Returns $0$ if
successful, a $-1$ is returned and an error message printed if no
LinearSOE object has been set in \p theAlgo. \\


\indent {\em int start(void);} \\
Sets an integer indicating the current number of iterations, {\em
currentNumIter} to $1$. returns $0$ if successfull, an error message
and $-1$ are returned if no LinearSOE object has been set. \\

\indent {\em int test(void);} \\
Returns {currentNumIter} if 0.5 times the absolute value of the dot product of
the LinearSOE objects X and B Vectors is less than the tolerance {\em
tol}. If no LinearSOE has been set $-2$ is returned. If the {\em
currentNumIter} $>=$ \p maxNumIter an error message is printed and
$-2$ is returned. If none of these conditions is met, the {\em
currentnumIter} is incremented and $-1$ is returned. If the print flag
is $0$ nothing is printed to 
opserr during the method, if $1$ the current iteration and norm are
printed to opserr, and if $2$ the norm and number of iterations to convergence
are printed to opserr. \\ 


\indent {\em int sendSelf(int commitTag, Channel \&theChannel);}\\ 
Creates a Vector of size 3, puts the tolerance value \p tol, {\em
numIncr} and \p printFlag
in this, and then invokes sendVector() on {\em
theChannel}. Returns $0$ if successful. A warning message is printed
and a negative number if the Channel object fails to send the Vector.\\ 

\indent {\em int recvSelf(int commitTag, Channel \&theChannel,
FEM\_ObjectBroker \&theBroker);}\\ 
Creates a Vector of size 3, invokes recvVector() on {\em
theChannel}, and sets the values of \p tol, \p numIncr and \p printFlag.
Returns $0$ if successful. If the Channel object fails to receive the
Vector, \p tol is set to $1.0e-8$, \p numIter to $25$, a warning
message is printed, and a negative number returned.y




