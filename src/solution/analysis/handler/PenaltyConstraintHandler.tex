%File: ~/OOP/analysis/handler/PenaltyConstraintHandler.tex
%What: "@(#) PenaltyConstraintHandler.tex, revA"

\noindent {\bf Files}
\indent \#include \f$<\tilde{
}\f$/analysis/handler/PenaltyConstraintHandler.h\f$>\f$ 

\noindent {\bf Class Declaration}
\indent class PenaltyConstraintHandler: public ConstraintHandler

\noindent {\bf Class Hierarchy}
\indent MovableObject
\indent\indent ConstraintHandler
\indent\indent\indent {\bf PenaltyConstraintHandler}

\noindent {\bf Description} 
\indent The PenaltyConstraintHandler class is a class which deals with
//! both single and multi point constraints using the penalty method. 
//! This is done by, in addition to creating a DOF\_Group object
//! for each Node and an FE\_Element for each Element in the Domain,
//! creating either a PenaltySFreedom\_FE or a PenaltyMP\_FE object for each
//! constraint in the Domain. It is these objects that enforce the
//! constraints by modifying the tangent matrix and residual vector. 


\noindent {\bf Class Interface}
\indent // Constructor
\indent {\em PenaltyConstraintHandler(double alphaSP, double
//! alphaMP);} 
\indent // Destructor
\indent {\em virtual~ \f$\tilde{}\f$PenaltyConstraintHandler();}
\indent // Public Methods
\indent {\em virtual int handle(const ID *nodeToBeNumberedLast
=0); } 
\indent {\em virtual void clearAll(void);}
\indent {\em int sendSelf(int commitTag, Channel \&theChannel); }
\indent {\em int recvSelf(int commitTag, Channel \&theChannel, FEM\_ObjectBroker
\&theBroker); }


\noindent {\bf Constructor}
\indent {\em PenaltyConstraintHandler(double alphaSp, double alphaMP);} 
//! The integer {\em HANDLER\_TAG\_PenaltyConstraintHandler} (defined in
\f$<\f$classTags.h\f$>\f$) is passed to the PenaltyConstraintHandler
//! constructor. Stores \p alphaSP and \p alphaMP which are needed
//! to construct the PenaltySFreedom\_FE and PenaltyMP\_FE objects in {\em
//! handle()}.

\noindent {\bf Destructor}
\indent {\em virtual~ \f$\tilde{}\f$PenaltyConstraintHandler();} 
//! Currently invokes clearAll(), this will change when {\em
//! clearAll()} is rewritten.

\noindent {\bf Public Methods }
\indent {\em virtual int handle(const ID *nodeToBeNumberedLast =0) =0;}
//! Determines the number of FE\_Elements and DOF\_Groups needed from the
//! Domain (a one to one mapping between Elements and FE\_Elements,
//! SFreedom\_Constraints and PenaltySFreedom\_FEs, MP\_Constraints and PenaltyMP\_FEs and
//! Nodes and DOF\_Groups). Creates two arrays of pointers to store the
//! FE\_Elements and DOF\_Groups, returning a warning message and a \f$-2\f$
//! or \f$-3\f$ if not enough memory is available for these arrays. Then the
//! object will iterate through the Nodes of the Domain, creating a
//! DOF\_Group for each node and setting the initial id for each dof to
\f$-2\f$ or \f$-3\f$ if the node identifier is in {\em
//! nodesToBeNumberedLast}. The object then iterates through the Elements
//! of the Domain creating a FE\_Element for each Element, if the Element
//! is a Subdomain setFE\_ElementPtr() is invoked on the Subdomain
//! with the new FE\_Element as the argument. If not enough memory is
//! available for any DOF\_Group or FE\_element a warning message is
//! printed and a \f$-4\f$ or \f$-5\f$ is returned. 
//! The object then iterates through the SFreedom\_Constraints
//! of the Domain creating a PenaltySFreedom\_FE for each constraint, using the
//! Domain, the constraint and \p alphaSP as the arguments in the
//! constructor.
//! The object then iterates through the MP\_Constraints
//! of the Domain creating a PenaltyMP\_FE for each constraint, using the
//! Domain, the constraint and \p alphaMP as the arguments in the constructor.
//! Finally the method returns the
//! number of degrees-of-freedom associated with the DOF\_Groups in {\em
//! nodesToBeNumberedLast}.

{\em virtual void clearAll(void) =0;}
//! Currently this invokes delete on all the FE\_element and DOF\_Group
//! objects created in handle() and the arrays used to store
//! pointers to these objects. FOR ANALYSIS INVOLVING DYNAMIC LOAD
//! BALANCING, RE-MESHING AND CONTACT THIS MUST CHANGE.

{\em int sendSelf(int commitTag, Channel \&theChannel); }
//! Sends in a Vector of size 2 \p alphaSP and \p alphaMP. Returns
\f$0\f$ if successful.

{\em int recvSelf(int commitTag, Channel \&theChannel, FEM\_ObjectBroker
\&theBroker); }
//! Receives in a Vector of size 2 the values \p alphaSP and {\em
//! alphaMP}. Returns \f$0\f$ if successful. 
