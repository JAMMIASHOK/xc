%File ~/OOP/analysis/algorithm/ModifiedNewton.tex
%What: "@(#) ModifiedNewton.tex, revA"

\noindent {\bf Files}
\indent \#include \f$<\tilde{
}\f$/analysis/algorithm/equiSolnAlgo/ModifiedNewton.h\f$>\f$ 

\noindent {\bf Class Declaration}
\indent class ModifiedNewton: public EquiSolnAlg;

\noindent {\bf Class Hierarchy}
\indent MovableObject
\indent\indent SolutionAlgorithm
\indent\indent\indent EquiSolnAlgo
\indent\indent\indent\indent {\bf ModifiedNewton}

\noindent {\bf Description} 
\indent The ModifiedNewton class is an algorithmic class which obtains a
//! solution to a non-linear system using the modified Newton-Raphson iteration
//! scheme. The Newton-Rapson iteration scheme is based on a Taylor expansion of the
//! non-linear system of equations \f$\R(\U) = \zero\f$ about an approximate
//! solution \f$\U{(i)}\f$:
\begin{equation} 
\R(\U) = 
\R(\U^{(i)}) +
\left[ {\frac{\partial \R}{\partial \U} \vert}_{\U^{(i)}}\right]
\left( \U - \U^{(i)} \right) 
\end{equation}

\noindent which can be expressed as:
\begin{equation} \
\K^{(i)}  \Delta \U{(i)} = \R(\U^{(i)})
\end{equation}
//! which is solved for \f$\Delta \U^{(i)}\f$ to give approximation for
\f$\U^{(i+1)} = \U^{(i)} + \Delta \U^{(i)}\f$. To start the
//! iteration \f$\U^{(1)} = \U_{trial}\f$, i.e. the current trial
//! response quantities are chosen as initial response quantities.

//! in the modified version the tangent is formed only once, i.e
\begin{equation} \
\K^{(1)}  \Delta \U^{(i)} = \R(\U^{(i)})
\end{equation}

//! To stop the iteration, a test must be performed to see if convergence
//! has been achieved at each iteration. Each NewtonRaphson object is
//! associated with a ConvergenceTest object. It is this object which
//! determines if convergence has been achieved.

\noindent {\bf Class Interface} 
\indent // Constructors 
\indent {\em ModifiedNewton(ConvergenceTest \&theTest);} 
\indent {\em ModifiedNewton();}
\indent // Destructor
\indent {\em ~ \f$\tilde{}\f$ModifiedNewton();}
\indent // Public Member Functions
\indent {\em int solveCurrentStep(void);}
\indent {\em void setTest(ConvergenceTest \&theTest);}
\indent // Public Methods  for Output
\indent {\em int sendSelf(int commitTag, Channel \&theChannel);} 
\indent {\em int recvSelf(int commitTag, Channel \&theChannel, 
//! FEM\_ObjectBroker \&theBroker);} 
\indent {\em int Print(OPS\_Stream \&s, int flag =0);}


\noindent {\bf Constructors} 
\indent {\em ModifiedNewton(int theMaxNumIter, ConvergenceTest \&theTest);} 
//! The constructor takes as an argument the ConvregenceTest object {\em
//! theTest}, the object which is used at the end of each iteration to
//! determine if convergence has been obtained. The
//! integer {\em EquiALGORITHM\_TAGS\_ModifiedNewton} (defined in
\f$<\f$classTags.h\f$>\f$) is passed to the EquiSolnAlgo classes
//! constructor. 

\indent {\em ModifiedNewton();}
//! Provided for FEM\_ObjectBroker to instantiate a blank object with a
//! class tag of EquiALGORITHM\_TAGS\_ModofiedNewton. {\em
//! recvSelf()} must be invoked on this object.

\noindent {\bf Destructor}
\indent {\em ~ \f$\tilde{}\f$ModifiedNewton();} 
//! Does nothing.

\noindent {\bf Public Member Functions}
\indent {\em int solveCurrentStep(void);}
//! When invoked the object first sets itself as the EquiSolnAlgo object
//! that the ConvergenceTest is testing and then it performs the
//! modified Newton-Raphson iteration algorithm: 
\begin{tabbing}
//! while \= \+ while \= \kill
//! theTest-\f$>\f$start();
//! theIntegrator-\f$>\f$formTangent();
//! do \{ \+
//! theIntegrator-\f$>\f$formUnbalance();
//! theSOE-\f$>\f$solveX();
//! theIntegrator-\f$>\f$update(theSOE-\f$>\f$getX()); \-
\} while (theTest-\f$>\f$test() \f$==\f$ false)\-
\end{tabbing}


\noindent The method returns a 0 if successful, otherwise a negative number is
//! returned; a \f$-1\f$ if error during formTangent(), a \f$-2\f$ if
//! error during formUnbalance(), a \f$-3\f$ if error during {\em
//! solve()}, and a \f$-4\f$ if error during update().
//! If an error occurs in any of the above operations the method stops at
//! that routine, none of the subsequent operations are invoked. A \f$-5\f$ is
//! returned if any one of the links has not been setup.

{\em void setTest(ConvergenceTest \&theTest);}
//! A method to set the tolerance criteria of the algorithm to be equal to
//! the value \p theTol.

{\em int sendSelf(int commitTag, Channel \&theChannel);}
//! Creates an ID object, puts the values of the \p theTest objects
//! class and database tags into this ID.
//! It then invokes sendVector() on the Channel object {\em
//! theChannel} to send the data to the remote object. It then invokes
//! sendSelf() on \p theTest. Returns \f$0\f$ if successful, the
 channel error if not.

{\em int recvSelf(int commitTag, Channel \&theChannel, FEM\_ObjectBroker
\&theBroker);} 
//! Creates an ID object, invokes recvVector() on the Channel
//! object. Uses the data in the ID to create a ConvergenceTest object of
//! appropriate type and sets its dbTag. It then invokes recvSelf()
//! on this test object.

{\em int Print(OPS\_Stream \&s, int flag =0);}
//! Sends the string 'ModifiedNewton' to the stream if \p flag equals \f$0\f$.
