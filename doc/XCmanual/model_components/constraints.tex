\section{Boundary conditions}
In a finite element problem, the boundary conditions\footnote{The following explanation is based on the \href{http://www.colorado.edu/engineering/CAS/courses.d/IFEM.d/}{IFEM courses} of the Department of Aerospace Engineering Sciences
University of Colorado at Boulder.} are the specified values of the field variables (displacement, rotations, pore pressures,\ldots).

\subsection{Essential and natural boundary conditions}
Essential boundary conditions are conditions that are imposed explicitly on the solution and natural boundary conditions are those that automatically will be satisfied after solution of the problem. Otherwise stated if boundary condition directly involves he nodal freedoms, such as displacements or rotations, it is essential.

This class of boundary conditions involve one or more degrees of freedom and are imposed by manipulating the left hand side (LHS) of the system of equations (the side of the stiffness matrix).

The natural boundary conditions are imposed by manipulating the right hand side (RHS) of the system of equations (the side of the force vector). Conditions involving applied loads are natural. This kind of constraints are treated in chapter \ref{ch_loads}.

\subsection{Constraints}

\subsubsection{Classification of Constraint Conditions}
In the previous  description we have said that an essential boundary condition can involve one or more degrees of freedom.

\paragraph{Single-freedom constraints.}
When there is only one condition involved we call them \emph{single-freedom constraints}. These conditions are mathematically expressable as constraints on individual degrees of freedom:

\begin{center}
  \fbox{nodal degree of fredom= prescribed value}
\end{center}

\noindent For example:

\begin{equation}
  u_{x4}= 0, u_{y9}= 0.6
\end{equation}

\noindent These are two single-freedom constraints.  The first one is homogeneous while the second one is non-homogeneous.

\paragraph{Multi-freedom constraints}
The next step up in complexity involves multifreedom equality constraints, or multifreedom constraints for short,  the last name being acronymed to MFC. These are functional equations that connect two or more displacement components:

\begin{center}
  \fbox{f(nodal degrees of fredom)= prescribed value}
\end{center}

\noindent or with a more formal mathematical notation:

\begin{equation}\label{eq_multi_freedom_constraints}
f(u_{x4}, u_{y9}, u_{y109})= p
\end{equation}

Equation \ref{eq_multi_freedom_constraints}, in which all displacement components are in the left-hand side, is called the canonical form of the constraint.

An MFC of this form is called \emph{multipoint} or \emph{multinode} if it involves displacement components at different nodes.  The constraint is called \emph{linear} if all displacement components appear linearly on the left-hand-side, and \emph{nonlinear} otherwise

The constraint is called \emph{homogeneous} if, upon transfering all terms that depend on displacement components to the left-hand side, the right-hand side — the ``prescribed value'' in (8.3) — is zero. It is called \emph{non-homogeneous} otherwise.

\subsubsection{Methods for imposing the constraints}
The methods for imposing the constraints are described in \ref{sc_constraint_handlers}.

\subsubsection{Single-point constraints.} \label{sc_sp_constraints}
A single-point constraint (SPC) enforces a single degree of freedom (normally associated with a node) to a specified value. In structural analysis we use single-point constraints to:

\begin{itemize}
  \item Constrain or enforce translations and/or rotations of nodes that correspond to structure supports.
  \item Impose constraints for displacements that correspond to symmetric or antisymmetric boundary conditions.
  \item Removal of DOFs that correspond to frozen nodes in evolutive problems wher some parts of the mesh are deactivated.
\end{itemize}


\subsection{MP constraints}

\subsubsection{Description}
An MP\_Constraint represents a multiple point constraint in the domain. A multiple point constraint imposes a relationship between the displacement for certain dof at two nodes in the model, typically called the {\em retained} node and the {\em constrained} node:

\begin{equation}
U_c = C_{cr} U_r
\end{equation}


An MP\_Constraint is responsible for providing information on the relationship between the dof, this is in the form of a constraint matrix, $C_{cr}$, and two ID objects, {\em retainedID} and {\em constrainedID} indicating the dof's at the nodes represented by $C_{cr}$. For example, for the following constraint imposing a relationship between the displacements at node $1$, the constrained node, with the displacements at node $2$, the retainednode in a problem where the x,y,z components are identified as the 0,1,2 degrees-of-freedom:

\begin{align}
u_{1,x} &= 2 u_{2,x} + u_{2,z} \\
u_{1,y} &= 3 u_{2,z}
\end{align}

\noindent the constraint matrix is:

\begin{equation}
C_{cr} =
\left[
\begin{array}{cc}
2 & 1  \\
0 & 3
\end{array}
\right] 
\end{equation}

\noindent and the vectors defining the dof's at the nodes are:

\begin{align}
constrainedID &= [0, 1] \\
retainedID &= [0, 2]
\end{align}



%%See
%% /documentacion/informatica/software/desarrollo/documentacion/elementos_finitos/felippa_ifem/IFEM.Ch08.pdf
