\chapter{Elements}

\section{Zero-Length Elements}
\subsection{ZeroLength}
The ZeroLength class represents an element defined by two nodes at the same geometric location, hence it has zero length.
The nodes are connected by of uniaxial materials to represent the force-deformation relationship for the element. 

ZeroLength elements are constructed with a {\em tag} in a domain of {\em dimension} 1, 2, or 3, connected by nodes {\em Nd1} and {\em Nd2}. 
The vector $\vec{x}$ defines the local x-axis for the element and the vector $\vec{yp}$ lies in the local x-y plane for the element.  The local z-axis is the cross product between $\vec{x}$ and $\vec{yp}$, and the local y-axis is the cross product between the local z-axis and $\vec{x}$.

The force-deformation relationship for the element is given by a pointer {\em theMaterial} to a {\bf UniaxialMaterial} model acting in local {\em direction}.

The local {\em direction} is 0, 1, 2 for translation in the local x, y, z axes or 3, 4, 5 for rotation about the local x, y, z axes. 

\begin{verbatim}
mdlr=xc.ProblemaEF().getModelador
ZeroLengthElement=mdlr.getElementLoader.newElement("zero_length",
xc.ID([Nd1Tag,Nd2Tag]))
\end{verbatim}
\begin{paramFuncTable}
{\tt Nd1Tag,Nd2Tag} & tags of the nodes connected by the element\\
\end{paramFuncTable}
\begin{paramClassTable}
\ElementParam{}
\ElementZERODParam{}
\end{paramClassTable}

\begin{methodsTable}
\ElementMeth{}
\ElementZERODMeth{}
\end{methodsTable}

\subsection{ZeroLengthSection}



\subsection{ZeroLengthContact}


\subsection{ZeroLengthContact2D}


\subsection{ZeroLengthContact3D}

