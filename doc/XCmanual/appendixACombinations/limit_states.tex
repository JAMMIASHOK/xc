\section{Limit states} \label{sc_el}
They can be defined as those states beyond which the structure no longer fulfils the relevant design criteria.

The design of the structure will be right when:

\begin{enumerate}
\item it is verified that no ultimate limit state is exceeded for the design situations and load cases defined in \ref{sc_comb_elu}, and
\item it is verified that no serviceability limit state is exceeded under the design situations and load cases defined in \ref{sc_comb_els}.
\end{enumerate}

\subsection{Ultimate limit states}
They are states associated with collapse or with other similar forms of structural failure. They generally correspond to the maximum load-carrying resistance of a structure or structural member.

The following ultimate limit states shall be verified where they are relevant:
- 
failure caused by fatigue or other time-dependent effects.


\begin{enumerate}
\item loss of equilibrium of the structure or any part of it, considered as a rigid body;
\item failure by excessive deformation, transformation of the structure or any part of it into a mechanism, rupture, loss of stability of the structure or any part of it, including supports and foundations;
\item failure caused by fatigue or other time-dependent effects.
\end{enumerate}

\subsection{Serviceability limit states}
They can be defined as states that correspond to conditions beyond which specified service requirements for a
structure or structural member are no longer met. These service requirements can concern:


\begin{itemize}
\item functionality.
\item comfort.
\item durability.
\item aesthetics.
\end{itemize}

The verification of serviceability limit states should be based on criteria concerning the following aspects :
\begin{enumerate}
\item deformations that affect:
\begin{itemize}
\item the appearance,
\item the comfort of users, or
\item the functioning of the structure (including the functioning of machines or services),
\end{itemize}
or that cause damage to finishes or non-structural members;

\item vibrations 
\begin{itemize}
\item that cause discomfort to people, or
\item that limit the functional effectiveness of the structure;
\end{itemize}
\item damage that is likely to adversely affect
\begin{itemize}
\item the appearance,
\item the durability, or
\item the functioning of the structure.
\end{itemize}
\end{enumerate}
